\section{Oppgave 13}
Forklar hvordan man kan løse
\begin{equation*}
    \frac{\partial u}{\partial t}=\frac{\partial^{2} u}{\partial x^{2}}
\end{equation*}
med rand- og initialkrav:
\begin{equation*}
    u(0, t)=u(1, t)=0 \quad u(x, 0)=f(x)
\end{equation*}
numerisk.
\subsection*{Besvarelse}
Selvfølgelig, la oss gå gjennom den numeriske løsningen av denne partielle differensialligningen (PDE) på norsk.

Den gitte PDE-en er en standard varmeligning, som modellerer diffusjonen av varme over tid. Randbetingelsene spesifiserer at endepunktene av domenet holdes ved null temperatur, og den innledende tilstanden gir en starttemperaturfordeling.

En vanlig numerisk metode for å løse denne PDE-en er Crank-Nicolson-metoden, som er en finitt differansemetode. Ideen bak metoden er å diskretisere rom- og tidsdomenet og deretter tilnærme derivatene med differensligninger.

Crank-Nicolson-metoden bruker en trapesregel for å tilnærme integralet, og gir dermed en god balanse mellom nøyaktighet og beregningskompleksitet.

La oss begynne med en detaljert beskrivelse av metoden.

Først, diskretiser domenet. Velg en steglengde $h=1/N$ for et stort $N$ og la $x_j=jh$ for $j=0,1,\ldots,N$. På samme måte, velg et tidssteg $\tau=T/M$ for et stort $M$ og la $t_k=k\tau$ for $k=0,1,\ldots,M$. Funksjonen $u$ blir tilnærmet på disse diskrete punktene: $u(x_j,t_k) \approx U_j^k$.

Crank-Nicolson-metoden fører da til følgende ligning for $U_j^k$:

\begin{equation*}
\frac{U_j^{k+1} - U_j^k}{\tau} = \frac{1}{2} \left( \frac{U_{j-1}^k - 2U_j^k + U_{j+1}^k}{h^2} + \frac{U_{j-1}^{k+1} - 2U_j^{k+1} + U_{j+1}^{k+1}}{h^2} \right)
\end{equation*}

for $j = 1,2,\ldots,N-1$ og $k=0,1,\ldots,M-1$. Dette er et system av lineære ligninger for ukjente $U_j^{k+1}$, $j=1,2,\ldots,N-1$. Legg merke til at systemet for $k+1$ inneholder vilkår fra både gjeldende tidsnivå $k$ og neste tidsnivå $k+1$.

Randbetingelsene er gitt ved $U_0^k = U_N^k = 0$ for en hvilken som helst $k$. Den innledende tilstanden er gitt ved $U_j^0=f(x_j)$ for $j=0,1,\ldots,N$.

For å løse systemet, må du omorganisere vilkårene. La oss innføre parameteren $r=\frac{\tau}{h^2}$. Da kan systemet skrives som følger:

\begin{equation*}
    r U_{j-1}^{k+1} + 2(1+r) U_j^{k+1} - r U_{j+1}^{k+1} = r U_{j-1}^k + 2(1-r) U_j^k + r U_{j+1}^k
\end{equation*}
for $j = 1,2,\ldots,N-1$. Disse ligningene danner et tridiagonalt system som kan løses ved hjelp av Thomas-algoritmen, som er en forenklet form for Gaussisk eliminering tilpasset tridiagonale systemer.

Denne prosessen gjentas for $k=0,1,\ldots,M-1$ for å løse alle $U_j^k$.

Så i oppsummering består løsningsprosessen av disse trinnene:

Diskretiser domenet til et rutenett.
På hvert tidspunkt, dann et system av ligninger basert på Crank-Nicolson-metoden.
Løs systemet av ligninger ved hjelp av Thomas-algoritmen.
Gjenta for det neste tidspunktet.
Det endelige resultatet er en tilnærming til løsningen $u(x,t)$ på hele rutenettet. Crank-Nicolson-metoden er betingelsesløst stabil og gir en løsning med andre ordens nøyaktighet, noe som betyr at det er en robust og effektiv metode for å løse denne typen PDE.