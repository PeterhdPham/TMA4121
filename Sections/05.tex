\section{Oppgave 5}
Skriv opp schrödingerlikningen og
\begin{enumerate}
    \item vis at det romlige arealet under kvadratet av løsningen er en invariant: 
    \begin{equation*}  
        \frac{d}{d t} \iiint_{\mathbb{R}^{3}}|\Psi(\mathbf{x}, t)|^{2} d \mathbf{x}=0    
    \end{equation*}
    \item separer variable og utled den tidsuavhengige schödingerlikningen
    \begin{equation*} 
        E \psi=-\frac{\hbar^{2}}{2 m} \Delta \psi+V \psi    
    \end{equation*}
\end{enumerate}

\subsection*{Besvarelse}
\subsubsection*{Løsningen er en incel}

Det er viktig å merke seg at det forenkles litt ved å anta at potensialet ikke er tidsavhengig og at partikkelen er ikke-relativistisk.

Schrödingerlikningen for en enkelt, ikke-relativistisk partikkel i et tidsuavhengig potensial er gitt ved:

\begin{equation*}
    i\hbar\frac{\partial}{\partial t}\Psi(\mathbf{x},t) = -\frac{\hbar^2}{2m}\nabla^2\Psi(\mathbf{x},t) + V(\mathbf{x})\Psi(\mathbf{x},t)
\end{equation*}

For å gjøre det lettere i framtiden så skriver jeg den om til 
\begin{equation*}
    \frac{\partial}{\partial t}\Psi(\mathbf{x},t) = -\frac{\hbar}{2mi}\nabla^2\Psi(\mathbf{x},t) + \frac{V(\mathbf{x})\Psi(\mathbf{x},t)}{i\hbar}
\end{equation*}

Vi ønsker å vise at:

\begin{equation*}
    \frac{d}{d t} \iiint_{\mathbb{R}^{3}}|\Psi(\mathbf{x}, t)|^{2} d \mathbf{x}=0
\end{equation*}

For å gjøre dette, tar vi tidsderiverte av dette uttrykket og bruker Schrödingerlikningen for å erstatte $\frac{\partial \Psi}{\partial t}$.

La oss først beregne den komplekskonjugerte av Schrödingerlikningen. Dette gir oss:

\begin{equation*}
    \frac{\partial}{\partial t}\Psi^*(\mathbf{x},t) = \frac{\hbar}{2mi}\nabla^2\Psi^*(\mathbf{x},t) - \frac{V(\mathbf{x})\Psi^*(\mathbf{x},t)}{i\hbar}
\end{equation*}

Vi kan nå uttrykke tidsderiverte av den romlige integrasjonen:

\begin{equation*}
    \frac{d}{dt} \iiint_{\mathbb{R}^{3}}|\Psi(\mathbf{x}, t)|^{2} d\mathbf{x} = \iiint_{\mathbb{R}^{3}} \frac{\partial}{\partial t} (\Psi^* \Psi) d\mathbf{x}
\end{equation*}

Her bruker vi produktregelen for derivasjon for å forenkle uttrykket:

\begin{equation*}
    \iiint_{\mathbb{R}^{3}} \frac{\partial}{\partial t} \left(\Psi^* \Psi \right) d\mathbf{x} = \iiint_{\mathbb{R}^{3}} \Psi^* \frac{\partial \Psi}{\partial t} + \Psi \frac{\partial \Psi^*}{\partial t} d\mathbf{x}
\end{equation*}

Nå kan vi bruke Schrödingerlikningen og dens komplekskonjugerte for å erstatte tidsderivertene. Dette gir oss:

\begin{equation*}
    = \iiint_{\mathbb{R}^{3}} \Psi^* \left(-\frac{\hbar}{2mi}\nabla^2\Psi(\mathbf{x},t) + \frac{V(\mathbf{x})\Psi(\mathbf{x},t)}{i\hbar}\right) d\mathbf{x} + \iiint_{\mathbb{R}^{3}} \Psi \left(\frac{\hbar}{2mi}\nabla^2\Psi^*(\mathbf{x},t) - \frac{V(\mathbf{x})\Psi^*(\mathbf{x},t)}{i\hbar}\right) d\mathbf{x}
\end{equation*}

Ved å forenkle uttrykket: 

\begin{equation*}
    =\frac{\hbar}{2mi} \iiint_{\mathbb{R}^{3}} \Psi^* \left(\nabla^2\Psi(\mathbf{x},t) + \frac{V(\mathbf{x})\Psi(\mathbf{x},t)}{i\hbar}\right) d\mathbf{x} + \iiint_{\mathbb{R}^{3}} \Psi \left(\frac{\hbar}{2mi}\nabla^2\Psi^*(\mathbf{x},t) - \frac{V(\mathbf{x})\Psi^*(\mathbf{x},t)}{i\hbar}\right) d\mathbf{x}
\end{equation*}

og bruke integraleiendommene, kan vi konkludere:

\begin{equation*}
    \frac{d}{dt} \iiint_{\mathbb{R}^{3}}|\Psi(\mathbf{x}, t)|^{2} d\mathbf{x} = \frac{1}{i\hbar}\iiint_{\mathbb{R}^{3}} \Psi^* \left(-\frac{\hbar^2}{2m}\nabla^2\Psi + V\Psi\right) d\mathbf{x} - \frac{1}{i\hbar}\iiint_{\mathbb{R}^{3}} \Psi \left(-\frac{\hbar^2}{2m}\nabla^2\Psi^* + V\Psi^*\right) d\mathbf{x}
\end{equation*}
    
Vi kan nå omorganisere dette for å isolere Laplacian-uttrykkene:
    
\begin{equation*}
    = -\frac{\hbar}{2m}\iiint_{\mathbb{R}^{3}} \Psi^* \nabla^2\Psi d\mathbf{x} + \frac{\hbar}{2m}\iiint_{\mathbb{R}^{3}} \Psi \nabla^2\Psi^* d\mathbf{x}
\end{equation*}
    
Så bruker vi produktregelen for derivasjon for å forenkle Laplacian-uttrykkene, noe som gir oss:
    
\begin{equation*}
    = \iiint_{\mathbb{R}^{3}} [\nabla \cdot (\Psi^* \nabla \Psi) - \nabla \cdot (\Psi \nabla \Psi^*)] d\mathbf{x}
\end{equation*}
    
Og så bruker vi Gauss' divergensteorem for å konvertere volumintegralet til et overflateintegral:
    
\begin{equation*}
    = \oint_{S} (\Psi^* \nabla \Psi - \Psi \nabla \Psi^*) \cdot d\mathbf{S}
\end{equation*}
    
Hvis vi antar at bølgefunksjonen $\Psi(\mathbf{x}, t)$ og dens romlige derivater går mot null når $|\mathbf{x}| \to \infty$, blir overflateintegralet null, noe som viser at
    
\begin{equation*}
    \frac{d}{dt} \iiint_{\mathbb{R}^{3}}|\Psi(\mathbf{x}, t)|^{2} d\mathbf{x} = 0
\end{equation*}
    
Dette beviser at det romlige arealet under kvadratet av løsningen til Schrödingerlikningen er en invariant.

\subsubsection*{Den tidsuavhengige schödingerlikningen}
la oss fortsette fra tidsavhengige Schrödingerlikningen:

\begin{equation*}
i\hbar\frac{\partial}{\partial t}\Psi(\mathbf{x},t) = -\frac{\hbar^2}{2m}\nabla^2\Psi(\mathbf{x},t) + V(\mathbf{x})\Psi(\mathbf{x},t)
\end{equation*}

Vi kan se etter løsninger som kan separeres i romlige og tidsavhengige deler. Vi antar derfor at $\Psi(\mathbf{x},t) = \psi(\mathbf{x})T(t)$, hvor $\psi(\mathbf{x})$ er den romlige delen av bølgefunksjonen, og $T(t)$ er den tidsavhengige delen.

Sett inn dette i Schrödingerlikningen:

\begin{equation*}
i\hbar \psi(\mathbf{x})\frac{dT}{dt} = -\frac{\hbar^2}{2m}T(t)\nabla^2\psi(\mathbf{x}) + V(\mathbf{x})\psi(\mathbf{x})T(t)
\end{equation*}

Dele begge sider med $\psi(\mathbf{x})T(t)$ for å separere variablene:

\begin{equation*}
i\hbar\frac{1}{T(t)}\frac{dT}{dt} = -\frac{\hbar^2}{2m}\frac{1}{\psi(\mathbf{x})}\nabla^2\psi(\mathbf{x}) + V(\mathbf{x})
\end{equation*}

Høyresiden av denne ligningen avhenger bare av romlige variabler, mens venstresiden kun avhenger av tid. Derfor må begge sidene være konstante for å oppfylle ligningen for alle $\mathbf{x}$ og $t$. Vi kaller denne konstanten for energi $E$, som gir oss to ligninger:

\begin{equation*}
i\hbar\frac{1}{T(t)}\frac{dT}{dt} = E
\end{equation*}

\begin{equation*}
-\frac{\hbar^2}{2m}\frac{1}{\psi(\mathbf{x})}\nabla^2\psi(\mathbf{x}) + V(\mathbf{x}) = E
\end{equation*}

Vi kan omorganisere den andre ligningen for å utlede den tidsuavhengige Schrödingerlikningen:

\begin{equation*}
E \psi(\mathbf{x})=-\frac{\hbar^{2}}{2 m} \nabla^2 \psi(\mathbf{x})+V(\mathbf{x})\psi(\mathbf{x})
\end{equation*}

Dette er den tidsuavhengige Schrödingerlikningen som vi ønsket å utlede. Den gir oss energiegenverdier og tilhørende romlige bølgefunksjoner for et gitt potensial $V(\mathbf{x})$.



\begin{equation*}
    f(x,y)=x^2-y^2
\end{equation*}