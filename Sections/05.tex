\section{Oppgave 5}
Skriv opp schrödingerlikningen og
\begin{enumerate}
    \item vis at det romlige arealet under kvadratet av løsningen er en invariant: \\ \begin{equation*}  \frac{d}{d t} \iiint_{\mathbb{R}^{3}}^{\mathbb{I}}|\Psi(\mathbf{x}, t)|^{2} d \mathbf{x}=0    \end{equation*}
    \item separer variable og utled den tidsuavhengige schödingerlikningen \\ \begin{equation*} E \psi=-\frac{\hbar^{2}}{2 m} \Delta \psi+V \psi    \end{equation*}
\end{enumerate}

\subsection*{Besvarelse}

La oss først skrive opp Schrödingerlikningen, som er kjernen i kvantemekanikken. Den tidavhengige Schrödingerlikningen kan uttrykkes som

\begin{equation*}
i\hbar\frac{\partial}{\partial t} \Psi(\mathbf{x},t) = -\frac{\hbar^2}{2m}\nabla^2\Psi(\mathbf{x},t) + V(\mathbf{x})\Psi(\mathbf{x},t)
\end{equation*}

der $\Psi(\mathbf{x},t)$ er bølgefunksjonen, $i$ er den imaginære enheten, $\hbar$ er den reduserte Plancks konstant, $m$ er partikkelens masse, $\nabla^2$ er Laplace-operatoren (som representerer en andre ordens derivasjon med hensyn til romlige koordinater), og $V(\mathbf{x})$ er potensialet som fungerer på systemet.

La oss nå vise at det romlige arealet under kvadratet av løsningen er en invariant, det vil si at det er uavhengig av tid. Vi starter med å beregne den tidsderiverte av $\int|\Psi(\mathbf{x}, t)|^{2} d \mathbf{x}$:
\begin{equation*}
\frac{d}{d t} \iiint_{\mathbb{R}^{3}} |\Psi(\mathbf{x}, t)|^{2} d \mathbf{x} = \iiint_{\mathbb{R}^{3}} \frac{d}{dt} |\Psi(\mathbf{x}, t)|^{2} d \mathbf{x}
\end{equation*}

Bruker produktregelen for derivasjon, har vi

\begin{equation*}
= \iiint_{\mathbb{R}^{3}} 2 \Psi^(\mathbf{x}, t) \frac{\partial}{\partial t} \Psi(\mathbf{x}, t) d \mathbf{x}
\end{equation}

der $\Psi^*(\mathbf{x}, t)$ er den komplekskonjugerte av $\Psi(\mathbf{x}, t)$.

Nå bruker vi Schrödingerlikningen for å erstatte $\frac{\partial}{\partial t} \Psi(\mathbf{x}, t)$:

\begin{equation*}
= \iiint_{\mathbb{R}^{3}} \frac{2}{i \hbar} \Psi^(\mathbf{x}, t) \left[-\frac{\hbar^2}{2m}\nabla^2\Psi(\mathbf{x},t) + V(\mathbf{x})\Psi(\mathbf{x},t)\right] d \mathbf{x}
\end{equation}

Og dette kan omskrives som

\begin{equation*}
= -\frac{2}{i \hbar} \iiint_{\mathbb{R}^{3}} \Psi^(\mathbf{x}, t) \left[\frac{\hbar^2}{2m}\nabla^2\Psi(\mathbf{x},t) - V(\mathbf{x})\Psi(\mathbf{x},t)\right] d \mathbf{x}
\end{equation}

Ved å bruke Gauss' divergen

steorem, kan vi konvertere volumintegralet av en divergens til et overflateintegral. Men siden bølgefunksjonen og dens derivater forutsettes å falle av til null når vi går mot uendelig, blir overflateintegralet null. Dermed blir

\begin{equation*}
\frac{d}{d t} \iiint_{\mathbb{R}^{3}} |\Psi(\mathbf{x}, t)|^{2} d \mathbf{x} = 0
\end{equation*}

Dette viser at det romlige arealet under kvadratet av løsningen er en invariant.

For å utlede den tiduavhengige Schrödingerlikningen, antar vi at vi kan separere variablene i bølgefunksjonen $\Psi(\mathbf{x}, t)$ som et produkt av en romlig del $\psi(\mathbf{x})$ og en tidsdel $T(t)$. Det vil si
\begin{equation*}
\Psi(\mathbf{x}, t) = \psi(\mathbf{x}) T(t)
\end{equation*}

Erstatter vi dette i den tidavhengige Schrödingerlikningen, får vi

\begin{equation*}
i\hbar \psi(\mathbf{x}) \frac{dT(t)}{dt} = -\frac{\hbar^2}{2m}T(t)\nabla^2\psi(\mathbf{x}) + V(\mathbf{x})\psi(\mathbf{x}) T(t)
\end{equation*}

Dette kan vi skrive om til

\begin{equation*}
i\hbar \frac{1}{T(t)} \frac{dT(t)}{dt} = -\frac{\hbar^2}{2m}\frac{1}{\psi(\mathbf{x})}\nabla^2\psi(\mathbf{x}) + V(\mathbf{x})
\end{equation*}

Siden venstre side kun avhenger av tid og høyre side kun av romlige koordinater, må begge sidene være lik en konstant. Denne konstanten kaller vi energien $E$ til systemet. Vi får dermed to ligninger:

For tidsdelen får vi

\begin{equation*}
i\hbar \frac{1}{T(t)} \frac{dT(t)}{dt} = E
\end{equation*}

som gir en løsning for $T(t)$ av formen

\begin{equation*}
T(t) = e^{-iEt/\hbar}
\end{equation*}

For den romlige delen får vi

\begin{equation*}
-\frac{\hbar^2}{2m}\frac{1}{\psi(\mathbf{x})}\nabla^2\psi(\mathbf{x}) + V(\mathbf{x}) = E
\end{equation*}

som kan omskrives til den tiduavhengige Schrödingerlikningen

\begin{equation*}
E\psi(\mathbf{x}) = -\frac{\hbar^2}{2m}\nabla^2\psi(\mathbf{x}) + V(\mathbf{x})\psi(\mathbf{x})
\end{equation*}

Dette er Schrödingerlikningen som kun inneholder romlige vari

abler. Denne likningen brukes for å finne stasjonære tilstander i kvantemekanikken.