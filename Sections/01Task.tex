 \homeworkProblem{}
 Skriv opp Maxwells likninger på differensialform og utled

 \begin{enumerate}
    \item de korresponderende likningene på integralform,
    \item likingen for ladningskonservering, og 
    \item bølgelikningen ved homogent tilfelle ($\rho=0$og $J=0$)
 \end{enumerate}

Maxwells fire likninger i differensialform:
\\
\textbf{1. :}
\begin{equation*}
    \nabla \cdot \mathbf{E} = \frac{\rho}{\epsilon_0}
\end{equation*}

hvor $\mathbf{E}$ er det elektriske feltet, $\rho$ er ladningstettheten, $\epsilon_0$ er den elektriske konstanten (også kjent som friromspermittiviteten), og $\nabla \cdot$ er divergensoperatoren.
\\
\textbf{2. Gauss'lov for magnetiske felt:}

\begin{equation*}
    \nabla \cdot \mathbf{B} = 0
\end{equation*}

hvor $\mathbf{B}$ er det magnetiske feltet, og $\nabla \cdot$ er divergensoperatoren.
\\
\textbf{3. Faradays induksjonslov:}

\begin{equation*}
    \nabla \times \mathbf{E} = -\frac{\partial \mathbf{B}}{\partial t}
\end{equation*}

hvor $\mathbf{E}$ er det elektriske feltet, $\mathbf{B}$ er det magnetiske feltet, $t$ er tiden, $\nabla \times$ er curl-operatoren, og $\partial/\partial t$ er den partielle derivasjonen med hensyn til tiden.
\\
\textbf{4. Ampères lov med Maxwells korreksjon:}

\begin{equation*}
    c^{2} \nabla \times \mathbf{B}=\frac{\partial \mathbf{E}}{\partial t}+\frac{\mathbf{J}}{\epsilon_{0}}
\end{equation*}

hvor $\mathbf{B}$ er det magnetiske feltet, $\mathbf{J}$ er strømtettheten, $c$ er lyshastigheten i vakuum, $\epsilon_0$ er den elektriske konstanten, $t$ er tiden, $\nabla \times$ er curl-operatoren, og $\partial/\partial t$ er den partielle derivasjonen med hensyn til tiden.

\subsection*{Maxwell på integralform}
\subsubsection*{1. Gauss' lov for elektriske felt:}
\begin{itemize}
    \item Vi starter med Gauss' lov i differentialform:
\end{itemize}

\begin{equation*}
    \nabla \cdot \mathbf{E}=\frac{\rho}{\epsilon_0}
\end{equation*}

Vi ønsker å finne en integralform for denne likningen ved å integrere begge sider av likningen over en volum V. Vi bruker Gauss' divergensteorem, som sier at for et vektorfelt $\mathbf{F}$ og et volum V, har vi:

\begin{equation*}
     \int_{V} \nabla \cdot \mathbf{F} d V=\oint_{\partial V} \mathbf{F} \cdot d \mathbf{A} 
\end{equation*}

der $\partial V$ er overflaten til volumet V og $d\mathbf{A}$ er et infinitesimalt overflateelement som peker normalt ut fra overflaten.

Vi bruker dette teoremet på venstre side av Gauss' lov i differentialform, og setter $\mathbf{F} = \mathbf{E}$:

\begin{equation*}
    \int_{V} \nabla \cdot \mathbf{E} d V=\oint_{\partial V} \mathbf{E} \cdot d \mathbf{A} 
\end{equation*}

Vi kan nå erstatte venstre side av denne likningen med høyresiden av Gauss' lov i differentialform. Vi multipliserer begge sider av denne likningen med $\epsilon_0$:

\begin{equation*}
     \oint_{\partial V} \mathbf{E} \cdot d \mathbf{A}=\frac{1}{\epsilon_{0}} \int_{V} \rho d V 
\end{equation*}

Dette gir oss Gauss' lov i integralform:

\begin{equation*}
    \oint_{\partial V} \mathbf{E} \cdot d\mathbf{A} = \frac{Q_{enc}}{\epsilon_0}
\end{equation*}

hvor $Q_{enc}$ er den totale ladningen som er inneholdt i volumet V. Dette er den første av Maxwells fire ligninger i integralform.

Denne ligningen sier at den totale elektriske fluksen gjennom en lukket overflate er proporsjonal med den totale ladningen som er inneholdt i volumet som omgir overflaten.


\subsubsection*{2. Gauss' lov for magnetiske felt:}
For å utlede Gauss' lov for magnetiske felt på integralform fra Gauss' lov for magnetiske felt på differensialform, kan vi bruke Gauss' divergensteorem, som sier at:

\begin{equation*}
    \oint_{\partial V} \mathbf{E} \cdot d \mathbf{A}=\frac{1}{\epsilon_{0}} \int_{V} \rho d V 
\end{equation*}
hvor $\mathbf{F}$ er et vektorfelt, $\oint_S$ betyr integral over en lukket flate $S$, $\int_V$ betyr integral over volumet som er begrenset av $S$, $d\mathbf{S}$ er et element av overflaten $S$, og $dV$ er et element av volumet.

Vi kan anvende dette teoremet på Gauss' lov for magnetiske felt på differensialform, som er:

\begin{equation*}
    \nabla \cdot \mathbf{B}=0 
\end{equation*}

Dette betyr at divergensen av det magnetiske feltet er null overalt i rommet. La oss nå velge en vilkårlig lukket overflate $S$ som omslutter et volum $V$. Gauss' divergensteorem gir da:

\begin{equation*}
    \oint_{S} \mathbf{B} \cdot d \mathbf{S}=\int_{V}(\nabla \cdot \mathbf{B}) d V
\end{equation*}

Siden divergensen av det magnetiske feltet er null, blir høyresiden av likningen lik null. Vi kan derfor forenkle uttrykket til:

\begin{equation*}
    \oint_{S} \mathbf{B} \cdot d \mathbf{S}=0 
\end{equation*}
Dette er Gauss' lov for magnetiske felt på integralform. Den sier at det totale magnetiske fluksen som går ut av en vilkårlig lukket flate er lik null. Denne loven gir oss en viktig sammenheng mellom magnetiske feltlinjer som går ut av en lukket flate, og den totale magnetiske ladningen inne i denne flaten.
\subsubsection*{3. Faradays induksjonslov:}
Faradays induksjonslov i differentialform er gitt ved:

\begin{equation*}
    \nabla \times \mathbf{E}=-\frac{\partial \mathbf{B}}{\partial t}
\end{equation*}

Vi kan utlede Faradays induksjonslov på integralform ved å anvende Stokes' teorem på denne ligningen. Stokes' teorem sier at:

\begin{equation*}
    \oint_{C} \mathbf{F} \cdot d \mathbf{r}=\iint_{S}(\nabla \times \mathbf{F}) \cdot d \mathbf{S}
\end{equation*}

hvor $C$ er en lukket kurve, $S$ er en flate som avgrenses av $C$, $\mathbf{F}$ er et vektorfelt, $d\mathbf{r}$ er en differensiell forflytning på kurven $C$, og $d\mathbf{S}$ er en differensiell flateelementvektor på flaten $S$.

Vi anvender Stokes' teorem på Faradays induksjonslov i differentialform:

\begin{equation*}
    \oint_{C} \mathbf{E} \cdot d \mathbf{r}=-\iint_{S} \frac{\partial \mathbf{B}}{\partial t} \cdot d \mathbf{S}
\end{equation*}

hvor $C$ er en lukket kurve som avgrenser flaten $S$.

Vi kan tolke venstresiden av denne ligningen som et elektromotoriskt spenn (EMF) $\mathcal{E}$ langs kurven $C$:

\begin{equation*}
    \mathcal{E}=\oint_{C} \mathbf{E} \cdot d \mathbf{r}
\end{equation*}

Så vi kan skrive Faradays induksjonslov på integralform som:

\begin{equation*}
    \mathcal{E}=-\frac{d}{d t} \iint_{S} \mathbf{B} \cdot d \mathbf{S}
\end{equation*}

hvor $S$ er en flate som avgrenses av kurven $C$, og $d\mathbf{S}$ er en differensiell flateelementvektor på flaten $S$.

Dette betyr at EMF langs en lukket kurve $C$ er lik minus den tidsderiverte av magnetisk fluks $\Phi_B$ som går gjennom flaten $S$ som er avgrenset av kurven $C$.

\subsubsection*{4. Ampères lov med Maxwells korreksjon:}
Vi kan starte med å skrive ut høyresiden av Maxwells likning med korreksjon på differensialform:

\begin{equation*}
    c^{2} \nabla \times \mathbf{B}=\frac{\partial \mathbf{E}}{\partial t}+\frac{\mathbf{J}}{\epsilon_{0}}
\end{equation*}

Vi kan nå ta curlen av begge sider av denne ligningen, og bruke Stokes' teorem på venstresiden:

\begin{equation*}
    c^{2} \oint_{\partial A} \mathbf{B} \cdot d \mathbf{l}=\iint_{A} \frac{\partial \mathbf{E}}{\partial t} \cdot d \mathbf{S}+\frac{1}{\epsilon_{0}} \iint_{A} \mathbf{J} \cdot d \mathbf{S}
\end{equation*}

Her representerer $A$ en vilkårlig, lukket flate som er begrenset av randen $\partial A$, og $\mathbf{l}$ er en vektor som peker langs randen.

Nå kan vi bruke Ampères lov på venstresiden av likningen, og bytte ut integranden med curlen av strømtettheten:

\begin{equation*}
    c^{2} \int_{\partial \Sigma} \mathbf{B} \cdot d \mathbf{s}=\iint_{A} \frac{\partial \mathbf{E}}{\partial t} \cdot d \mathbf{S}+\frac{1}{\epsilon_{0}} \iint_{A} \nabla \times \mathbf{J} \cdot d \mathbf{S}
\end{equation*}

Her representerer $\Sigma$ en vilkårlig, lukket kurve som er begrenset av randen $\partial \Sigma$, og $\mathbf{s}$ er en vektor som peker langs kurven.

Til slutt kan vi bruke divergensteoremet på høyresiden av likningen, og bytte ut integranden med divergensen av strømtettheten:

\begin{equation*}
    c^{2} \int_{\partial \Sigma} \mathbf{B} \cdot d \mathbf{s}=\frac{1}{\epsilon_{0}} \iint_{\Sigma} \mathbf{J} \cdot d \mathbf{S}+\frac{d}{d t} \iint_{\Sigma} \mathbf{E} \cdot d \mathbf{S}
\end{equation*}

Dette er likningen for Ampères lov med Maxwells korreksjon på integralform. Den beskriver den magnetiske feltstyrken langs en lukket kurve som er proporsjonal med strømmen som passerer gjennom flaten som begrenses av kurven, samt den tidsavledede elektriske fluksen gjennom denne flaten.

\subsection*{Likningen for ladningskonservering}
Vi kan utlede likningen for ladningskonservering ved å bruke Maxwells likninger. Vi starter med Gauss' lov for elektromagnetiske felt:

\begin{equation*}
\nabla \cdot \mathbf{E} = \frac{\rho}{\epsilon_0}
\end{equation*}

der $\rho$ er ladningstettheten og $\epsilon_0$ er vakuumpermittiviteten. Vi tar deretter tidsderivaten av begge sider av denne ligningen:

\begin{equation*}
\frac{\partial}{\partial t}(\nabla \cdot \mathbf{E}) = \frac{1}{\epsilon_0}\frac{\partial \rho}{\partial t}
\end{equation*}

Vi kan nå bruke Faradays induksjonslov:

\begin{equation*}
\nabla \times \mathbf{E} = -\frac{\partial \mathbf{B}}{\partial t}
\end{equation*}

til å erstatte den tidsderiverte av divergensen av $\mathbf{E}$ på venstresiden av likningen over:

\begin{equation*}
\nabla \cdot (\frac{\partial \mathbf{E}}{\partial t} + \nabla \times \mathbf{B}) = \frac{1}{\epsilon_0}\frac{\partial \rho}{\partial t}
\end{equation*}

Her har vi brukt Maxwells likninger i differensialform. Vi kan nå bruke Gauss' lov for magnetiske felt:

\begin{equation*}
\nabla \cdot \mathbf{B} = 0
\end{equation*}

til å forenkle venstresiden av likningen til

\begin{equation*}
\nabla \cdot (\frac{\partial \mathbf{E}}{\partial t}) = \frac{1}{\epsilon_0}\frac{\partial \rho}{\partial t}
\end{equation*}

Vi bruker så divergens-setningen for å skrive om venstresiden av likningen:

\begin{equation*}
\int_V \nabla \cdot (\frac{\partial \mathbf{E}}{\partial t}) dV = \oint_{\partial V} \frac{\partial \mathbf{E}}{\partial t} \cdot d\mathbf{a}
\end{equation*}

der $V$ er et vilkårlig volum og $\partial V$ er overflaten av dette volumet.

Vi kan nå bruke Ladningsloven:

\begin{equation*}
\oint_{\partial V} \mathbf{J} \cdot d\mathbf{a} = \frac{d}{dt} \int_V \rho dV
\end{equation*}

til å erstatte høyresiden av likningen:

\begin{equation*}
\int_V \nabla \cdot (\frac{\partial \mathbf{E}}{\partial t}) dV = \frac{1}{\epsilon_0} \oint_{\partial V} \mathbf{J} \cdot d\mathbf{a} + \frac{d}{dt} \int_V \rho dV
\end{equation*}

Dette er nå likningen for ladningskonservering i integralform, og den kan brukes til å beskrive hvordan elektrisk ladning beveger segVi kan utlede likningen for ladningskonservering ved å bruke Maxwells likninger. Vi starter med Gauss' lov for elektromagnetiske felt:

\begin{equation*}
\nabla \cdot \mathbf{E} = \frac{\rho}{\epsilon_0}
\end{equation*}

der $\rho$ er ladningstettheten og $\epsilon_0$ er vakuumpermittiviteten. Vi tar deretter tidsderivaten av begge sider av denne ligningen:

\begin{equation*}
\frac{\partial}{\partial t}(\nabla \cdot \mathbf{E}) = \frac{1}{\epsilon_0}\frac{\partial \rho}{\partial t}
\end{equation*}

Vi kan nå bruke Faradays induksjonslov:

\begin{equation*}
\nabla \times \mathbf{E} = -\frac{\partial \mathbf{B}}{\partial t}
\end{equation*}

til å erstatte den tidsderiverte av divergensen av $\mathbf{E}$ på venstresiden av likningen over:

\begin{equation*}
\nabla \cdot (\frac{\partial \mathbf{E}}{\partial t} + \nabla \times \mathbf{B}) = \frac{1}{\epsilon_0}\frac{\partial \rho}{\partial t}
\end{equation*}

Her har vi brukt Maxwells likninger i differensialform. Vi kan nå bruke Gauss' lov for magnetiske felt:

\begin{equation*}
\nabla \cdot \mathbf{B} = 0
\end{equation*}

til å forenkle venstresiden av likningen til

\begin{equation*}
\nabla \cdot (\frac{\partial \mathbf{E}}{\partial t}) = \frac{1}{\epsilon_0}\frac{\partial \rho}{\partial t}
\end{equation*}

Vi bruker så divergens-setningen for å skrive om venstresiden av likningen:

\begin{equation*}
\int_V \nabla \cdot (\frac{\partial \mathbf{E}}{\partial t}) dV = \oint_{\partial V} \frac{\partial \mathbf{E}}{\partial t} \cdot d\mathbf{a}
\end{equation*}

der $V$ er et vilkårlig volum og $\partial V$ er overflaten av dette volumet.

Vi kan nå bruke Ladningsloven:

\begin{equation*}
\oint_{\partial V} \mathbf{J} \cdot d\mathbf{a} = \frac{d}{dt} \int_V \rho dV
\end{equation*}

til å erstatte høyresiden av likningen:

\begin{equation*}
\int_V \nabla \cdot (\frac{\partial \mathbf{E}}{\partial t}) dV = \frac{1}{\epsilon_0} \oint_{\partial V} \mathbf{J} \cdot d\mathbf{a} + \frac{d}{dt} \int_V \rho dV
\end{equation*}

Dette er nå likningen for ladningskonservering i integralform, og den kan brukes til å beskrive hvordan elektrisk ladning beveger segVi kan utlede likningen for ladningskonservering ved å bruke Maxwells likninger. Vi starter med Gauss' lov for elektromagnetiske felt:

\begin{equation*}
\nabla \cdot \mathbf{E} = \frac{\rho}{\epsilon_0}
\end{equation*}

der $\rho$ er ladningstettheten og $\epsilon_0$ er vakuumpermittiviteten. Vi tar deretter tidsderivaten av begge sider av denne ligningen:

\begin{equation*}
\frac{\partial}{\partial t}(\nabla \cdot \mathbf{E}) = \frac{1}{\epsilon_0}\frac{\partial \rho}{\partial t}
\end{equation*}

Vi kan nå bruke Faradays induksjonslov:

\begin{equation*}
\nabla \times \mathbf{E} = -\frac{\partial \mathbf{B}}{\partial t}
\end{equation*}

til å erstatte den tidsderiverte av divergensen av $\mathbf{E}$ på venstresiden av likningen over:

\begin{equation*}
\nabla \cdot (\frac{\partial \mathbf{E}}{\partial t} + \nabla \times \mathbf{B}) = \frac{1}{\epsilon_0}\frac{\partial \rho}{\partial t}
\end{equation*}

Her har vi brukt Maxwells likninger i differensialform. Vi kan nå bruke Gauss' lov for magnetiske felt:

\begin{equation*}
\nabla \cdot \mathbf{B} = 0
\end{equation*}

til å forenkle venstresiden av likningen til

\begin{equation*}
\nabla \cdot (\frac{\partial \mathbf{E}}{\partial t}) = \frac{1}{\epsilon_0}\frac{\partial \rho}{\partial t}
\end{equation*}

Vi bruker så divergens-setningen for å skrive om venstresiden av likningen:

\begin{equation*}
\int_V \nabla \cdot (\frac{\partial \mathbf{E}}{\partial t}) dV = \oint_{\partial V} \frac{\partial \mathbf{E}}{\partial t} \cdot d\mathbf{a}
\end{equation*}

der $V$ er et vilkårlig volum og $\partial V$ er overflaten av dette volumet.

Vi kan nå bruke Ladningsloven:

\begin{equation*}
\oint_{\partial V} \mathbf{J} \cdot d\mathbf{a} = \frac{d}{dt} \int_V \rho dV
\end{equation*}

til å erstatte høyresiden av likningen:

\begin{equation*}
\int_V \nabla \cdot (\frac{\partial \mathbf{E}}{\partial t}) dV = \frac{1}{\epsilon_0} \oint_{\partial V} \mathbf{J} \cdot d\mathbf{a} + \frac{d}{dt} \int_V \rho dV
\end{equation*}

Dette er nå likningen for ladningskonservering i integralform, og den kan brukes til å beskrive hvordan elektrisk ladning beveger seg og samhandler i elektromagnetiske systemer.

\subsection*{Bølgelikningen ved homogent tilfelle}
Maxwells likninger i den homogene tilfellet med $\rho=0$ og $\mathbf{J}=0$ er:

\begin{align*}
\nabla \cdot \mathbf{E} &= 0 \\
\nabla \cdot \mathbf{B} &= 0 \\      
\nabla \times \mathbf{E} &= -\frac{\partial \mathbf{B}}{\partial t} \\           
\nabla \times \mathbf{B} &= \mu_0\epsilon_0\frac{\partial \mathbf{E}}{\partial t}
\end{align*}

Vi tar curl på likningen $\nabla \times \mathbf{E} = -\frac{\partial \mathbf{B}}{\partial t}$:

\begin{equation*}
\nabla \times (\nabla \times \mathbf{E}) = -\frac{\partial}{\partial t} (\nabla \times \mathbf{B})
\end{equation*}

Ved å bruke identiteten for curl av curl og Faradays lov $\nabla \times \mathbf{E} = -\frac{\partial \mathbf{B}}{\partial t}$, får vi:

\begin{equation*}
\nabla(\nabla \cdot \mathbf{E})-\nabla^2\mathbf{E} = -\frac{\partial}{\partial t}(\mu_0\epsilon_0\frac{\partial \mathbf{E}}{\partial t})
\end{equation*}

Siden $\nabla \cdot \mathbf{E} = 0$ for homogene tilfellet, reduseres likningen til:

\begin{equation*}
\nabla^2\mathbf{E} = \mu_0\epsilon_0\frac{\partial^2 \mathbf{E}}{\partial t^2}
\end{equation*}

Dette er bølgelikningen for det elektromagnetiske feltet $\mathbf{E}$ i den homogene tilfellet.