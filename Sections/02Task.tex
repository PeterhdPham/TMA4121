\homeworkProblem[2]

\subsection*{Kvantebrønn}

Kvantebrønner er en modell som beskriver oppførselen til partikler som elektroner i potensialbrønner på kvantemekanisk nivå.

Figur: tegn en kvantebrønn.

\subsection*{Uendeleg djup kvantebrønn i 1D}
Vi starter med en uendelig dyp kvantebrønn.

figur: tegn en brønn.

Denne brønnen er en forenklet modell der potensialbrønnen er uendelig dyp og partikler kan ikke eksistere utenfor brønnen.

Schrödinger-ligningen for en partikkel i en én-dimensjonal uendelig dyp kvantebrønn er:

\begin{equation*}
-\frac{\hbar^2}{2m}\frac{d^2\psi(x)}{dx^2} = E\psi(x),
\end{equation*}

hvor $\hbar$ er Plancks reduserte konstant, $m$ er partikkelens masse, $\psi(x)$ er bølgefunksjonen og $E$ er energien til partikkelen.

Vi kan også skrive:

\begin{equation*}
    \psi'' = \frac{2mE}{\hbar}, \text{  der, } k^2 = \frac{2mE}{\hbar^2} 
\end{equation*}

Vi ser at den dobbelderiverte av bølgefunksjonen gir oss en negativ konstant foran.

\begin{equation*}
    \psi(x)=A\cos(kx)+B\sin(kx)
\end{equation*}

med grensebetingelsene $\psi(0)=\psi(L)=0$

får vi at:

\begin{equation*}
    \psi(x)=\sqrt[root]{\frac{2}{L}}\sin(\frac{n\pi x}{L})
\end{equation*}


grunnet grensebetingelsene har vi at.

\begin{equation*}
E_n = \frac{n^2\pi^2\hbar^2}{2mL^2},
\end{equation*}

hvor $n$ er en positiv heltall og $L$ er brønnens bredde.

tegn brønn med flere nivåer.

\subsection*{Endeleg kvantebrønn}
En endelig kvantebrønn er en mer realistisk modell enn en uendelig dyp kvantebrønn, der potensialet i brønnen er endelig og partikler kan eksistere utenfor brønnen, men med en lavere sannsynlighet. Dette fører til at både bundne og spredningstilstander kan eksistere i en endelig kvantebrønn.

Schrödinger-ligningen for en partikkel i en én-dimensjonal endelig kvantebrønn er den samme som for en uendelig dyp kvantebrønn, men med en endret potensialfunksjon $V(x)$:

\begin{equation*}
-\frac{\hbar^2}{2m}\frac{d^2\psi(x)}{dx^2} + V(x)\psi(x) = E\psi(x),
\end{equation*}

En viktig forskjell mellom endelig og uendelig dyp kvantebrønn er at bølgefunksjonene i en endelig kvantebrønn ikke er begrenset til å være null ved brønnens kanter og kan ha en liten, men ikke-null verdi utenfor brønnen. Dette betyr at elektronene i en endelig kvantebrønn har en viss sannsynlighet for å bli funnet utenfor brønnen.

For å illustrere en endelig kvantebrønn, kan du tegne en figur som viser potensialbrønnen og bølgefunksjonene til de forskjellige energinivåene, inkludert hvordan bølgefunksjonene strekker seg utenfor brønnen.

\subsection*{Bundne og spreiande tilstandar}
Bundne og spredningstilstander er løsninger av Schrödinger-ligningen avhengig av potensialet og partikkelens energi. Bundne tilstander har diskrete energinivåer og partikler begrenset innenfor et område, mens spredningstilstander har kontinuerlige energinivåer og partikler som beveger seg fritt. I en uendelig dyp kvantebrønn er alle tilstander bundne, og det er ingen spredningstilstander. Disse tilstandene påvirker elektriske og optiske egenskaper i faste stoffer, med bundne tilstander relatert til elektroner i atomer, molekyler eller krystaller, og spredningstilstander relatert til ledningselektroner som bidrar til elektrisk ledningsevne. For å illustrere dette, tegn en figur som viser potensialbrønnen og bølgefunksjonene til de forskjellige energinivåene.

\subsection*{Tunnellering}
Kvantetunnellering er et fenomen der partikler, som elektroner, kan passere gjennom en potensialbarriere som de normalt ikke ville kunne gjøre i klassisk mekanikk. Dette fenomenet er relevant for kvantebrønner fordi det tillater elektroner å bevege seg fra en brønn til en annen gjennom en potensialbarriere, noe som har viktige konsekvenser for faststoffelektronikk.

Tunnellering i en endelig kvantebrønn kan analyseres ved å løse Schrödinger-ligningen for partikkelen inne i og utenfor potensialbarrieren. For en partikkel som befinner seg utenfor barrieren, vil bølgefunksjonen ha en eksponentiell form som representerer den avtagende sannsynligheten for å finne partikkelen utenfor brønnen:

\begin{equation*}
\psi(x) = Ae^{-\kappa x},
\end{equation*}

hvor $\kappa = \sqrt{\frac{2m(V_0 - E)}{\hbar^2}}$

\subsection*{Perturbasjonsteori}
Perturbasjonsteori er en metode for å finne energinivåer og bølgefunksjoner i kvantebrønner med små forstyrrelser i potensialet, der den eksakte løsningen av Schrödinger-ligningen er vanskelig eller umulig å finne. Perturbasjonsteori tar utgangspunkt i en kjent løsning og beregner endringene som skyldes forstyrrelsen.

For å anvende perturbasjonsteori, skriv Schrödinger-ligningen som:

\begin{equation*}
(\hat{H}_0 + \hat{V}')\psi(x) = E\psi(x),
\end{equation*}

der $\hat{H}_0$ er den kjente Hamilton-operatøren uten forstyrrelsen, $\hat{V}'$ er forstyrrelsen og $\psi(x)$ og $E$ er bølgefunksjonen og energien til partikkelen, henholdsvis.

Perturbasjonsteori benytter seg av en styrkeparameter $\lambda$ for å beskrive forholdet mellom forstyrrelsen og den opprinnelige potensialbrønnen:

\begin{equation*}
\hat{V}' = \lambda\hat{V}.
\end{equation*}

Ved å ekspandere bølgefunksjonen og energien i en potensrekke i $\lambda$, kan man finne approksimerte løsninger for de korrigerte energinivåene og bølgefunksjonene.

Perturbasjonsteori har mange anvendelser i faststoffelektronikk, for eksempel i beregning av energinivåene og bølgefunksjonene i krystaller med defekter, eller i systemer med eksterne elektriske eller magnetiske felt.

\subsection*{Målepostulatet}
Målepostulatet er et grunnleggende prinsipp i kvantemekanikk som beskriver hvordan målinger av partikkelens posisjon, energi og andre egenskaper utføres. Målepostulatet sier at når en måling utføres på en partikkel, vil partikkelen bli funnet i en bestemt tilstand som er beskrevet av en egenfunksjon til den målte operator og dens respektive egenverdi.

For måling av elektronposisjon og energi i en kvantebrønn, kan man benytte seg av posisjons- og energioperatorer:

\begin{equation*}
\hat{x}\psi(x) = x\psi(x),
\end{equation*}

\begin{equation*}
\hat{H}\psi(x) = E\psi(x),
\end{equation*}

hvor $\hat{x}$ er posisjonsoperatøren og $\hat{H}$ er Hamilton-operatøren.

Usikkerhetsprinsippet, formulert av Heisenberg, setter en fundamental grense på nøyaktigheten av målinger av partiklers posisjon og bevegelsesmengde (og dermed energi). Dette prinsippet er spesielt viktig i kvantebrønner, der partiklene er begrenset i et lite område og har kvantemekaniske egenskaper som bølge-partikkel-dualitet.

Målepostulatet og usikkerhetsprinsippet har praktiske implikasjoner for faststoffelektronikk, for eksempel i utformingen av nanoskala enheter og systemer, hvor kvanteeffekter blir mer fremtredende og må tas i betraktning for å forstå og styre elektroniske egenskaper.

Under presentasjonen kan du vurdere å tegne en figur som viser en kvantebrønn med en partikkel, og deretter vise hvordan målepostulatet og usikkerhetsprinsippet påvirker partikkelens posisjon og energi.

\begin{equation*}
    \left[-\frac{\hbar^2}{2m} \frac{\partial^2}{\partial x^2}+U(x)\right]\psi(X)=E\psi(x)
\end{equation*}

