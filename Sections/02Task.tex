\homeworkProblem[2]

\subsection*{Kvantebrønn}

Kvantebrønner er en modell som beskriver oppførselen til partikler som elektroner i potensialbrønner på kvantemekanisk nivå.

\subsection*{Uendeleg djup kvantebrønn i 1D}
En uendelig dyp kvantebrønn er en modell der et partikkel befinner seg i et potensial som er null innenfor brønnen og uendelig høyt utenfor. I én dimensjon kan vi betrakte partikkelen som beveger seg langs en x-akse, hvor brønnen strekker seg fra x = 0 til x = a.

I denne modellen kan vi løse Schrödinger-ligningen for å finne energinivåene og bølgefunksjonene til partikkelen. Siden potensialet er uendelig høyt utenfor brønnen, må bølgefunksjonen være null der. Innenfor brønnen må bølgefunksjonen og dens andrederiverte være kontinuerlig. Siden den ikke kan eksistere utenfor brønnen så impiserer dette også at bølgelikningen er normaliserbar.

Ved å løse Schrödinger-ligningen for denne situasjonen, finner vi at energinivåene er kvantiserte og gitt ved formelen:
\begin{equation*}
E_n = \frac{n^2\pi^2\hbar^2}{8ma^2},
\end{equation*}
hvor $n$ er en positiv heltall og $a$ er brønnens bredde.

Bølgefunksjonene for bundne tilstander har formen:
\begin{equation*}
    \psi_n(X)=\sqrt{\frac{2}{a}}\cdot \sin (\frac{n \pi x}{a})
\end{equation*}
der n er et positivt heltall, og x er posisjonen innenfor brønnen.

\subsection*{Bundne og spreiande tilstandar}
Bundne tilstander er de tilstandene som er innesperret i brønnen, og de er karakterisert ved diskrete energinivåer og bølgefunksjoner som går mot null utenfor brønnen. Spreiande tilstander, derimot, er ikke begrenset til brønnen og har kontinuerlige energinivåer. I en uendelig dyp kvantebrønn er det ingen spreiande tilstander, siden partikkelen ikke kan eksistere utenfor brønnen.

\subsection*{Endeleg kvantebrønn}
En endelig kvantebrønn er en modell der et partikkel befinner seg i et potensial som er null innenfor brønnen og en konstant, $V_0$, utenfor brønnen. Akkurat som i det uendelig dype tilfellet, kan vi betrakte partikkelen som beveger seg langs en x-akse, hvor brønnen strekker seg fra $x = 0$ til $x = a$.

Sammenlignet med den uendelig dype kvantebrønnen, har endelige kvantebrønner noen viktige forskjeller i energinivåer og bølgefunksjoner. For det første er ikke energinivåene strengt kvantiserte. Det vil fortsatt være diskrete energinivåer for bundne tilstander, men det er også mulig for partikkelen å ha spreiande tilstander med energier større enn $V_0$.

For å finne energinivåene og bølgefunksjonene i en endelig kvantebrønn, må vi løse Schrödinger-ligningen separat for de tre områdene: x < 0, 0 < x < L og x > L. Dette er mer komplisert enn for den uendelig dype brønnen, og løsningene vil involvere både trigonometriske funksjoner og eksponentialfunksjoner.

Innenfor brønnen (0 < x < L), vil bølgefunksjonen ha en form som likner den for den uendelig dype brønnen:
\begin{equation*}
\psi(x) = A \sin(kx) + B \cos(kx),
\end{equation*}
der A og B er konstanter som bestemmes av randbetingelsene, og k er bølgetallet, gitt ved:
\begin{equation*}
k = \sqrt{\frac{2mE}{\hbar^2}}.
\end{equation*}
Utenfor brønnen (x < 0 og x > L) vil bølgefunksjonen ha en eksponentiell form, som representerer den eksponentielle avtagningen av sannsynligheten for å finne partikkelen utenfor brønnen:
\begin{equation*}
\psi(x) = C e^{-\alpha x} + D e^{\alpha x},
\end{equation*}
der C og D er konstanter, og $\alpha$ er gitt ved:
\begin{equation*}
    \alpha = \sqrt{\frac{2m(V_0 - E)}{\hbar^2}}.
\end{equation*}
\subsection*{Perturbasjonsteori}
Dersom vi introduserer en liten endring i potensiale neddi brønnen så kan vi bruke perturbasjonsteori, som er en viktig metode for å løse det kvantemekaniske problemet. 

Perturbasjonsteori innebærer å betrakte potensialet som en sum av en kjent del og en liten perturbasjon. Vi kan skrive potensialet som:
\begin{equation*}
V(x) = V_0(x) + \lambda V_1(x),
\end{equation*}
der $V_0(x)$ er det kjente potensialet, for eksempel en endelig kvantebrønn, $V_1(x)$ er perturbasjonen, og $\lambda$ er en liten parameter som angir styrken av perturbasjonen.

Vi antar at vi kjenner energinivåene og bølgefunksjonene for $V_0(x)$, og vi ønsker å finne hvordan de endres på grunn av perturbasjonen. Vi utvider energiene og bølgefunksjonene i serier i $\lambda$:
\begin{equation*}
E_n = E_n^{(0)} + \lambda E_n^{(1)} + \lambda^2 E_n^{(2)} + \cdots,
\end{equation*}
\begin{equation*}
\psi_n(x) = \psi_n^{(0)}(x) + \lambda \psi_n^{(1)}(x) + \lambda^2 \psi_n^{(2)}(x) + \cdots,
\end{equation*}
der $E_n^{(0)}$ og $\psi_n^{(0)}(x)$ er energiene og bølgefunksjonene for $V_0(x)$, og $E_n^{(1)}$, $\psi_n^{(1)}(x)$, $E_n^{(2)}$, $\psi_n^{(2)}(x)$, osv. er korreksjoner på grunn av perturbasjonen.

Vi kan deretter sette disse utvidelsene inn i Schrödinger-ligningen og samle ledd med samme potens av $\lambda$. Dette gir oss et sett med hierarkiske ligninger som vi kan løse for å finne korreksjonene til energiene og bølgefunksjonene.

\subsection*{Tunnellering}
Tunnellering oppstår når en partikkel har en ikke-null sannsynlighet for å befinne seg på den andre siden av en potensiell barriere, selv om dens energi er lavere enn barrierens høyde.

For å forstå tunnellering, kan vi tenke på en endelig kvantebrønn med en partikkel som har en energi E som er lavere enn høyden på den potensielle barrieren, $V_0$. Bølgefunksjonen for partikkelen ha en eksponentiell form utenfor brønnen, noe som betyr at det er en ikke-null sannsynlighet for å finne partikkelen utenfor brønnen, selv om dens energi er lavere enn barrierens høyde.

Sannsynligheten for tunnellering avhenger av både høyden og bredden på barrieren, samt energien til partikkelen. Generelt vil sannsynligheten for tunnellering avta eksponentielt med økende bredde og økende forskjell mellom barrierens høyde og partikkelens energi.

\subsection*{Målepostulatet}

For å kunne koble den matematiske beskrivelsen av kvantetilstander til fysiske målinger og observasjoner kan vi ta i bruk prinsippet målepostulatet.

I tilfellet med en kvantebrønn, kan vi bruke målepostulatet til å forutsi sannsynlighetsfordelingen for å finne partikkelen på ulike steder innenfor og utenfor brønnen.

Målepostulatet består av to hovedprinsipper:

Forventningsverdien av en observabel, for eksempel posisjon eller energi, kan finnes ved å ta det indre produktet av bølgefunksjonen og den tilsvarende operatorrepresentasjonen av observabelen:

\begin{equation*}
    \langle A \rangle = \int \psi^{(x)} \hat{A} \psi(x) dx,
\end{equation*}

der $\psi^*(x)$ er den komplekskonjugerte bølgefunksjonen, $\hat{A}$ er operatoren som representerer observabelen, og $\psi(x)$ er bølgefunksjonen.

Når en måling utføres, vil systemet kollapse til en av egenfunksjonene til den målte observabelen med en sannsynlighet som er proporsjonal med absoluttkvadratet av bølgefunksjonen. Etter målingen vil systemet være i denne egenfunksjonen med en bestemt verdi av observabelen. Dette kan også likne veldig på dobbeltspalteeksperimentet.