\homeworkProblem[6]

\subsection*{Båndstruktur i 1. Brillouinsone}

I dette avsnittet vil jeg forklare begrepet Brillouin-soner og deres betydning for båndstrukturen. En Brillouin-sone er et område i gjensidig rom, som representerer de forskjellige bølgevektorene (k-vektorer) for elektronene i et krystallinsk materiale. Den første Brillouin-sonen er den minste og enkleste av Brillouin-sonene og er avgrenset av Bragg-planene.

For å forstå hvorfor Brillouin-soner er viktige for båndstrukturen, la oss først se på krystallstrukturen i et fast stoff. Krystallstrukturen i et fast stoff er et periodisk arrangement av atomer som danner et gitter. Gitteret er beskrevet av gittervektorene, som er de minste vektorene som forbinder gitterpunktene.

Når elektronene i et krystallinsk materiale beveger seg gjennom gitteret, vil de oppleve en periodisk potensial som påvirker deres bølgefunksjon. Dette resulterer i at elektronenes energinivåer blir samlet i bånd, som danner båndstrukturen. Båndstrukturen gir en detaljert beskrivelse av elektronenes energinivåer og bevegelser i et materiale, og er derfor viktig for å forstå de elektriske og optiske egenskapene til materialet.

Den første Brillouin-sonen er spesielt viktig for båndstrukturen fordi den inneholder de mest relevante k-vektorene for elektronene i materialet. Disse k-vektorene bestemmer elektronenes energinivåer og bidrar til dannelsen av energibåndene. Siden den første Brillouin-sonen inneholder de laveste energinivåene, er den mest relevant for å forstå de grunnleggende egenskapene til materialet, som for eksempel ledningsevne og optiske egenskaper.

For å visualisere båndstrukturen i den første Brillouin-sonen, kan vi plotte energinivåene som funksjon av k-vektorene. Dette gir oss et bånddiagram, som viser energibåndene og deres avhengighet av k-vektorene. Ved å analysere bånddiagrammet kan vi få verdifull innsikt i materialets egenskaper, som for eksempel ledningsevne, båndgap og optiske egenskaper.

\subsection*{Tilstandstettleik (DOS)}

Tilstandstetthet, eller Density of States (DOS), er et viktig konsept i forståelsen av båndstrukturen og elektriske egenskaper til et materiale. Tilstandstetthet beskriver antallet tilgjengelige energinivåer for elektroner per energienhetsintervall i et gitt materiale. Med andre ord, den angir hvor tett energinivåene er pakket sammen i materialets båndstruktur.

For å beregne tilstandstettheten, kan vi bruke følgende generelle formel:

\begin{equation*}
D(E) = \frac{dN}{dE}
\end{equation*}

hvor $D(E)$ er tilstandstettheten ved energi $E$, og $dN$ og $dE$ er henholdsvis antallet tilstander og energiintervallet. Denne formelen kan bli mer komplisert avhengig av materialets båndstruktur og dimensjonalitet, men hovedkonseptet er det samme.

Ved å analysere tilstandstettheten, kan vi få innsikt i hvor sannsynlig det er for elektroner å befinne seg på bestemte energinivåer. For eksempel vil et høyt DOS-verdi ved en gitt energi bety at det er mange tilgjengelige energinivåer for elektroner, og dermed øker sannsynligheten for at elektroner vil befinne seg på disse nivåene. Dette er spesielt viktig når vi vurderer ledningsevne og optiske egenskaper til et materiale, da disse egenskapene er sterkt påvirket av elektronenes energinivåer og deres sannsynlighet for å bli okkupert.

For eksempel, i halvledere vil tilstandstettheten i valensbåndet og ledningsbåndet være avgjørende for å forstå hvordan elektroner og hull bidrar til materialets ledningsevne. En høy DOS i valensbåndet betyr at det er mange tilgjengelige energinivåer for elektroner, noe som gjør det lettere for dem å bli eksitert og skape hull. På samme måte vil en høy DOS i ledningsbåndet bety at det er mange ledige energinivåer for elektroner som er eksitert fra valensbåndet, noe som øker ledningsevnen.

Tilstandstettheten kan bli visualisert ved å plotte DOS som funksjon av energi. Dette gir oss et DOS-diagram, som gir oss et visuelt bilde av hvordan energinivåene er distribuert i materialet. Ved å sammenligne DOS-diagrammer for forskjellige materialer, kan vi få innsikt i deres relative elektriske og optiske egenskaper og dermed velge det mest passende materialet for en gitt anvendelse.

\subsection*{Fermi-Dirac fordelinga}
Fermi-Dirac-fordelingen er et annet sentralt konsept for å forstå båndstrukturen og de elektriske egenskapene til et materiale. Fermi-Dirac-fordelingen beskriver sannsynligheten for at en gitt energitilstand er okkupert av et elektron ved en bestemt temperatur. Den er spesielt viktig for å forstå hvordan elektroner og hull bidrar til ledningsevnen i et materiale, siden den gir oss informasjon om hvilke energinivåer som er mest sannsynlig å bli okkupert av elektroner.

Fermi-Dirac-fordelingen er gitt ved følgende formel:

\begin{equation*}
f(E) = \frac{1}{e^{\frac{E - E_F}{k_B T}} + 1}
\end{equation*}

hvor $f(E)$ er sannsynligheten for at en energitilstand med energi $E$ er okkupert av et elektron, $E_F$ er Fermi-energien, $k_B$ er Boltzmanns konstant, og $T$ er temperaturen.

Fermi-energien er et viktig konsept i forståelsen av Fermi-Dirac-fordelingen og båndstrukturen. Det er den energien ved hvilken sannsynligheten for at en energitilstand er okkupert av et elektron er 50 % ved absolutt null temperatur ($T = 0$ K). Fermi-energien spiller en viktig rolle i å bestemme materialets ledningsevne, da den angir energinivået der elektroner og hull begynner å bli dannet.

Temperaturen påvirker også Fermi-Dirac-fordelingen, da høyere temperaturer vil resultere i at flere elektroner blir eksitert til høyere energinivåer, og dermed øke ledningsevnen. Dette er spesielt viktig for halvledermaterialer, hvor ledningsevnen er sterkt temperaturavhengig.

Ved å analysere Fermi-Dirac-fordelingen i forbindelse med båndstrukturen og tilstandstettheten, kan vi få en mer fullstendig forståelse av hvordan elektroner og hull bidrar til materialets ledningsevne og optiske egenskaper. For eksempel, i et halvledermateriale kan vi bruke Fermi-Dirac-fordelingen for å finne sannsynligheten for at et bestemt energinivå i valensbåndet er okkupert av et elektron og dermed bidrar til dannelsen av hull.

Fermi-Dirac-fordelingen kan visualiseres ved å plotte sannsynligheten for okkupasjon som funksjon av energi. Dette gir oss et bilde av hvordan elektronene er distribuert over energinivåene i materialet og hjelper oss med å forstå hvordan de elektriske og optiske egenskapene til materialet påvirkes av temperatur og Fermi-energi.

\subsection*{Elektron og hull i båndstrukturen}

Elektroner og hull er to fundamentale konsepter i båndstrukturen som er avgjørende for å forstå materialets elektriske egenskaper, spesielt i halvledere. I dette avsnittet vil jeg forklare begrepene elektroner og hull og hvordan de oppstår i båndstrukturen.

Elektroner er negativt ladede partikler som befinner seg i energibåndene i et materiale. I isolatorer og halvledere er elektronene vanligvis konsentrert i valensbåndet, som er det høyeste energibåndet som er helt fylt med elektroner ved absolutt null temperatur. Når elektroner får tilført energi, for eksempel fra termisk agitasjon eller lysabsorpsjon, kan de bli eksitert fra valensbåndet til ledningsbåndet, som er det neste høyeste energibåndet over valensbåndet.

Hull er en positivt ladet "mangel" på elektroner i valensbåndet som oppstår når elektroner blir eksitert til ledningsbåndet. Hull kan tenkes på som de tomme plassene som elektronene etterlater seg i valensbåndet når de blir eksitert. Selv om hull ikke er fysiske partikler, kan de behandles som kvasepartikler som bidrar til materialets ledningsevne.

Ledningsevnen i et materiale er i stor grad bestemt av antallet elektroner og hull og deres mobilitet. I metaller, hvor valensbåndet og ledningsbåndet overlapper, er det et stort antall elektroner som er fritt tilgjengelige for å delta i ledning. I halvledere, derimot, er det en energigap mellom valensbåndet og ledningsbåndet, og dermed færre elektroner og hull tilgjengelige for å lede strøm.

\subsection*{Båndstruktur i 3D}
Så langt har vi diskutert båndstrukturen i en forenklet, eendimensjonal form. Imidlertid er krystallinske materialer vanligvis tredimensjonale, og det er derfor viktig å forstå hvordan båndstrukturen utvider seg til 3D. I en tredimensjonal båndstruktur vil energibåndene og k-vektorene (bølgevektorene) bli påvirket av materialets krystallstruktur og symmetri i alle tre dimensjoner.

I en 3D-båndstruktur er det vanlig å beskrive energibåndene som funksjon av k-vektorene i det gjensidige rommet, som er en tredimensjonal Fourier-transformasjon av det virkelige rommet. For å visualisere båndstrukturen i 3D, kan vi plotte energinivåene som funksjon av k-vektorene langs høytsymmetriretningene i det gjensidige rommet. Dette gir oss et 3D-bånddiagram som viser hvordan energibåndene varierer i de tre dimensjonene.

3D-båndstrukturen gir en mer komplett beskrivelse av elektronenes energinivåer og bevegelse i et materiale, og er derfor viktig for å forstå materialets elektriske og optiske egenskaper. For eksempel kan en tredimensjonal analyse av båndstrukturen gi oss innsikt i hvordan elektronenes mobilitet og ledningsevnen i materialet påvirkes av krystallstruktur og symmetri. I tillegg kan 3D-båndstrukturen hjelpe oss med å identifisere og forstå eksotiske kvantefenomener som topologiske isolatorer og Weyl-semimetaller.

Det er verdt å merke seg at beregning og visualisering av 3D-båndstrukturen kan være mer komplisert og tidkrevende enn i 1D-tilfellet. Det finnes imidlertid en rekke verktøy og teknikker, både eksperimentelle og teoretiske, som kan hjelpe oss med å studere 3D-båndstrukturen i materialer, for eksempel elektronisk båndstrukturberegning ved hjelp av tetthetsfunksjonell teori (DFT) og målinger av elektronisk struktur ved hjelp av fotoemisjonsspektroskopi.

\subsection*{Direkte og indirekte båndgap}

Når vi snakker om båndgap i materialer, spesielt i halvledere, er det viktig å skille mellom direkte og indirekte båndgap. Dette skillet har betydning for materialets optiske og elektroniske egenskaper og påvirker dets anvendelser i ulike teknologier, som f.eks. solceller og lysdioder (LEDs).

Et direkte båndgap er en situasjon der det laveste energinivået i ledningsbåndet og det høyeste energinivået i valensbåndet har samme k-vektor (bølgevektor) i det gjensidige rommet. I et materiale med direkte båndgap kan elektroner overgå direkte mellom valensbåndet og ledningsbåndet ved å absorbere eller emitere fotoner uten å endre k-vektoren betydelig. Dette gjør at direkte båndgap-materialer er effektive lysabsorbenter og -emittere, og derfor er mye brukt i optoelektroniske applikasjoner som solceller og lysdioder.

Et indirekte båndgap er en situasjon der det laveste energinivået i ledningsbåndet og det høyeste energinivået i valensbåndet har forskjellige k-vektorer i det gjensidige rommet. I et materiale med indirekte båndgap må elektroner endre både energi og k-vektor for å overgå mellom valensbåndet og ledningsbåndet. Dette innebærer at de må samhandle med fotoner og fononer (krystallgittervibrasjoner) samtidig. Denne prosessen er mindre sannsynlig enn den direkte overgangen, og derfor er indirekte båndgap-materialer generelt mindre effektive som lysabsorbenter og -emittere.

Valget mellom direkte og indirekte båndgap-materialer avhenger av den spesifikke applikasjonen og de ønskede egenskapene. For eksempel er direkte båndgap-materialer som galliumarsenid (GaAs) og indiumfosfid (InP) mye brukt i høyhastighets elektronikk og optoelektronikk på grunn av deres gode lysabsorpsjon og -emisjonsegenskaper. På den annen side er indirekte båndgap-materialer som silisium (Si) og germanium (Ge) mye brukt i elektronikk på grunn av deres gode elektriske egenskaper og tilgjengelighet, selv om deres optiske egenskaper er mindre gunstige sammenlignet med direkte båndgap-materialer.

\subsection*{Forenkla bånddiagram}

Et forenklet bånddiagram er en skjematisk representasjon av båndstrukturen til et materiale som fokuserer på hovedtrekkene som er viktige for å forstå materialets elektriske og optiske egenskaper. Forenkla bånddiagrammer brukes ofte for å illustrere konsepter som båndgap, elektroner og hull, og direkte og indirekte overganger på en enkel og lettfattelig måte.

For å tegne et forenklet bånddiagram, kan du følge disse trinnene:


\begin{itemize}
    \item Tegn to horisontale linjer for å representere valensbåndet (nederste linje) og ledningsbåndet (øverste linje). Valensbåndet inneholder elektronene i de høyeste okkuperte energinivåene, mens ledningsbåndet inneholder de laveste ledige energinivåene.
    \item Angi båndgapet mellom valensbåndet og ledningsbåndet ved å tegne en loddrett avstand mellom de to linjene. Båndgapet representerer energiforskjellen mellom det høyeste okkuperte energinivået i valensbåndet og det laveste ledige energinivået i ledningsbåndet.
    \item Hvis det er relevant, merk om materialet har et direkte eller indirekte båndgap ved å plassere en pil som representerer en direkte overgang (samme k-vektor) eller en indirekte overgang (forskjellig k-vektor) mellom valensbåndet og ledningsbåndet.
    \item For å vise elektroner og hull i båndstrukturen, tegn små sirkler eller prikker i valensbåndet (elektroner) og ledningsbåndet (hull). Antall elektroner og hull kan variere avhengig av materialets type og dopingnivå.
    \item Hvis det er relevant, indiker Fermi-energien ved å tegne en horisontal stiplede linje som krysser bånddiagrammet. Fermi-energien gir informasjon om sannsynligheten for at en gitt energitilstand er okkupert av et elektron.
\end{itemize}

Et forenklet bånddiagram kan brukes for å illustrere og forklare mange viktige konsepter innen faststoffelektronikk, som f.eks. hvordan elektroner og hull bidrar til ledningsevnen i et materiale, hvordan direkte og indirekte båndgap påvirker materialets optiske egenskaper og hvordan temperaturen og doping påvirker Fermi-energien og ledningsevnen.