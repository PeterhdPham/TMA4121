\section{Oppgave 2}

Skriv opp Maxwells likninger på differensialform og utled Poissons likning ved statisk tilfelle $ \left(\frac{\partial \mathbf{E}}{\partial t}=\frac{\partial \mathbf{B}}{\partial t}=\mathbf{0}\right) $. Utled så potensialet til coulombfeltet i to og tre dimensjoner ved å lete etter harmoniske funksjoner som kun avhenger av $ \|\mathbf{x}\| $.

\subsection*{Besvarelse}

Maxwells likninger på differensialform er:

\begin{enumerate}
\item Gauss' lov for elektrisitet: $\nabla \cdot \mathbf{E} = \frac{\rho}{\epsilon_0}$.
\item Gauss' lov for magnetisme: $\nabla \cdot \mathbf{B} = 0$.
\item Faradays lov: $\nabla \times \mathbf{E} = -\frac{\partial \mathbf{B}}{\partial t}$.
\item Ampères lov med Maxwells tillegg: $c^2 \nabla \times \mathbf{B} = \frac{\partial \mathbf{E}}{\partial t} + \frac{\mathbf{J}}{\epsilon_0}$.
\end{enumerate}


Når vi nå betrakter det statiske tilfellet, der det ikke er tidsavhengighet ($\frac{\partial \mathbf{E}}{\partial t}=\frac{\partial \mathbf{B}}{\partial t}=\mathbf{0}$), kan vi forenkle Maxwells ligninger til

\begin{equation*}
    c^2 \nabla \times \mathbf{B} = \frac{\mathbf{J}}{\epsilon_0}
\end{equation*}
\begin{equation*}
\nabla \times \mathbf{E} = \mathbf{0}
\end{equation*}

Den andre ligningen impliserer at det elektriske feltet $\mathbf{E}$ er konservativt, noe som betyr at det kan uttrykkes som gradienten til et skalarpotensial $V$, dvs. $\mathbf{E} = - \nabla V$.

Ved å sette dette inn i den første ligningen får vi

\begin{equation*}
\nabla \cdot (-\nabla V) = \frac{\rho}{\varepsilon_0}
\end{equation*}

Eller, ved omforming får vi Poissons ligning i elektrostatikk:

\begin{equation*}
\nabla^2 V = - \frac{\rho}{\varepsilon_0}
\end{equation*}

Nå, for å finne potensialet for Coulomb-feltet, leter vi etter løsninger til Laplaces ligning i to og tre dimensjoner. Laplaces ligning er rett og slett Poissons ligning med $\rho = 0$:

\begin{equation*}
\nabla^2 V = 0
\end{equation*}

For sfærisk symmetri, i to dimensjoner, forenkles Laplace-operatoren i polarkoordinater ($r, \theta$) til:

\begin{equation*}
\nabla^2 V = \frac{\partial^2 V}{\partial r^2} + \frac{1}{r} \frac{\partial V}{\partial r}
\end{equation*}

Ved å sette dette lik null og løse den resulterende ordinære differensialligningen med betingelsen om at potensialet er endelig ved $r=0$, får vi den generelle løsningen

\begin{equation*}
V(r) = A \ln r + B
\end{equation*}

I tre dimensjoner er Laplace-operatoren i sfæriske koordinater ($r, \theta, \varphi$) gitt ved:

\begin{equation*}
\nabla^2 V = \frac{1}{r^2} \frac{\partial}{\partial r} \left(r^2 \frac{\partial V}{\partial r} \right)
\end{equation*}

Ved å sette dette lik null og løse den resulterende ordinære differensialligningen med betingelsen om at potensialet er endelig ved $r=0$, får vi den generelle løsningen

\begin{equation*}
V(r) = \frac{A}{r} + B
\end{equation*}

For en punktladning $Q$ som befinner seg i origo, er den elektriske ladningstettheten $\rho$ gitt ved $\rho(\mathbf{r}) = Q \delta(\mathbf{r})$, der $\delta(\mathbf{r})$ er Dirac-deltafunksjonen. Dette betyr at det elektriske feltet $\mathbf{E}(\mathbf{r})$ kun avhenger av den radielle koordinaten $r = |\mathbf{r}|$.

I dette tilfellet blir løsningen på Poissons ligning det elektriske potensialet for en punktladning, som er gitt av Coulombs lov.

I to dimensjoner har vi

\begin{equation*}
V(r) = A \ln r + B
\end{equation*}

Siden vi ønsker at potensialet skal gå mot null når $r \rightarrow \infty$, setter vi $B = 0$. Konstanten $A$ kan finnes ved å vurdere det elektriske feltet, som er den negative gradienten til potensialet: $\mathbf{E} = -\nabla V = -A/r \mathbf{\hat{r}}$. Ved å sette dette lik den kjente formen for Coulombs lov, $\mathbf{E} = kq/r^2 \mathbf{\hat{r}}$, får vi $A = -kQ$. Derfor blir potensialet i to dimensjoner

\begin{equation*}
V(r) = -kQ \ln r
\end{equation*}

\begin{equation*}
    V(r) = - \frac{Q}{2\pi \epsilon_0} \ln r
    \end{equation*}

hvor $k$ er Coulombs konstant $k=\frac{1}{4\varepsilon_0}$.

I tre dimensjoner har vi

\begin{equation*}
V(r) = \frac{A}{r} + B
\end{equation*}

Igjen setter vi $B = 0$ for å få potensialet til å forsvinne ved uendeligheten. Konstanten $A$ kan finnes på en lignende måte ved å sette det elektriske feltet, $\mathbf{E} = -\nabla V = -A/r^2 \mathbf{\hat{r}}$, lik Coulombs lov, for å få $A = kQ$. Derfor blir potensialet i tre dimensjoner.

\begin{equation*}
V(r) = \frac{Q}{4\pi \varepsilon_0 r}
\end{equation*}

her er $k=\frac{1}{4\varepsilon_0}$
\subsubsection*{Hvorfor vi får de generelle løsningene}
I to dimensjoner forenkles Laplace-operatoren i polarkoordinater ($r, \theta$) til:

\begin{equation*}
\nabla^2 \phi = \frac{\partial^2 \phi}{\partial r^2} + \frac{1}{r} \frac{\partial \phi}{\partial r}
\end{equation*}

Ved å sette dette lik null får vi:

\begin{equation*}
\frac{\partial^2 \phi}{\partial r^2} + \frac{1}{r} \frac{\partial \phi}{\partial r} = 0
\end{equation*}

Dette er en standardform for Bessels ligning, med løsninger på formen av Besselfunksjoner. Men siden vi leter etter en løsning som bare avhenger av $r$ og som er jevn for $r=0$, kan vi anta en potensrekkeløsning. Den eneste løsningen som ikke divergerer for $r=0$ er en konstant, men vi må også tillate en logaritmisk singularitet for $r=0$. Dette fører oss til løsningen:

\begin{equation*}
\phi(r) = A \ln r + B
\end{equation*}

der $A$ og $B$ er konstanter.

La oss nå gå videre til tre dimensjoner. Laplace-operatoren i sfæriske koordinater ($r, \theta, \varphi$) er:

\begin{equation*}
\nabla^2 \phi = \frac{1}{r^2} \frac{\partial}{\partial r} \left(r^2 \frac{\partial \phi}{\partial r} \right)
\end{equation*}

Ved å sette dette lik null får vi:

\begin{equation*}
\frac{1}{r^2} \frac{\partial}{\partial r} \left(r^2 \frac{\partial \phi}{\partial r} \right) = 0
\end{equation*}

Integrering én gang med hensyn til $r$ gir:

\begin{equation*}
\frac{\partial}{\partial r} \left(r^2 \frac{\partial \phi}{\partial r} \right) = C
\end{equation*}

der $C$ er en integrasjonskonstant. Ved omforming og integrering én gang til med hensyn til $r$ får vi:

\begin{equation*}
r^2 \frac{\partial \phi}{\partial r} = Cr + D
\end{equation*}

der $D$ er en annen integrasjonskonstant. Ved å dele på $r^2$ og integrere en siste gang, får vi:

\begin{equation*}
\phi(r) = \frac{A}{r} + B
\end{equation*}

der $A=C$ og $B$ er konstanter. Dette er løsningen til Laplaces ligning i tre dimensjoner for en funksjon som bare avhenger av $r$ og som er jevn for $r=0$.