\section{Oppgave 3}
Anta tomt rom og vis at dersom bølgefunksjonene  $\mathbf{E}(\mathbf{x}$, $t)=\mathbf{E}_{0} e^{i(\mathbf{k} \cdot \mathbf{x}-c t)}$  og  $\mathbf{B}(\mathbf{x}, t)=   \mathbf{B}_{0} e^{i(\mathbf{k} \cdot \mathbf{x}-c t)} $ skal tilfredsstille Maxwells likninger, må de konstante vektorene  $\mathbf{E}_{0}, \mathbf{B}_{0}$  og  $\mathbf{k}$  være innbyrdes ortogonale.

\subsubsection*{Besvarelse}

I dette tomrommet, hvor det ikke er noen ladninger eller strømmer, blir Maxwells likninger noe forenklet. De blir:

\begin{itemize}
    \item $\nabla \cdot \mathbf{E} = 0$.
    \item $\nabla \cdot \mathbf{B} = 0$.
    \item $\nabla \times \mathbf{E} = -\frac{\partial \mathbf{B}}{\partial t}$.
    \item $c^2 \nabla \times \mathbf{B} = \frac{\partial \mathbf{E}}{\partial t}$.
\end{itemize}

La oss starte med likning 1 og 2, Gauss' lov for elektrisitet og magnetisme, og se hva som skjer når vi setter inn våre bølgefunksjoner.

For elektrisitetsfeltet får vi:

\begin{equation*}
\nabla \cdot \mathbf{E} = \nabla \cdot (\mathbf{E}_0 e^{i(\mathbf{k} \cdot \mathbf{x}-ct)}) = i \mathbf{k} \cdot \mathbf{E}_0 e^{i(\mathbf{k} \cdot \mathbf{x}-ct)} = 0
\end{equation*}

Og for magnetfeltet:

\begin{equation*}
\nabla \cdot \mathbf{B} = \nabla \cdot (\mathbf{B}_0 e^{i(\mathbf{k} \cdot \mathbf{x}-ct)}) = i \mathbf{k} \cdot \mathbf{B}_0 e^{i(\mathbf{k} \cdot \mathbf{x}-ct)} = 0
\end{equation*}

Fra disse likningene ser vi at for at begge skal være tilfredsstilt, må vi ha $\mathbf{k} \cdot \mathbf{E}_0 = 0$ og $\mathbf{k} \cdot \mathbf{B}_0 = 0$. Dette betyr at både $\mathbf{E}_0$ og $\mathbf{B}_0$ er ortogonale til bølgevektoren $\mathbf{k}$.

Videre kan vi se på Faradays lov (likning 3) og Ampères lov (likning 4) for å finne forholdet mellom $\mathbf{E}_0$ og $\mathbf{B}_0$:

Fra Faradays lov får vi:

\begin{equation*}
\nabla \times \mathbf{E} = -\frac{\partial \mathbf{B}}{\partial t} \rightarrow i \mathbf{k} \times \mathbf{E}_0 e^{i(\mathbf{k} \cdot \mathbf{x}-ct)} = -i c \mathbf{B}_0 e^{i(\mathbf{k} \cdot \mathbf{x}-ct)}
\end{equation*}

Som kan forenkles til:

\begin{equation*}
\mathbf{k} \times \mathbf{E}_0 = c \mathbf{B}_0
\end{equation*}

Og fra Ampères lov får vi:
\begin{equation*}
    c^2 \nabla \times \mathbf{B} = \frac{\partial \mathbf{E}}{\partial t} \rightarrow c^2 i \mathbf{k} \times \mathbf{B}_0 e^{i(\mathbf{k} \cdot \mathbf{x}-ct)} = c \mathbf{E}_0 e^{i(\mathbf{k} \cdot \mathbf{x}-ct)}
\end{equation*}
    

Som kan forenkles til:

\begin{equation*}
c \mathbf{k} \times \mathbf{B}_0 = \mathbf{E}_0
\end{equation*}

Fra disse to siste likningene ser vi at $\mathbf{E}_0$ er ortogonal til $\mathbf{B}_0$, siden $\mathbf{E}_0$ er parallel med kryssproduktet av $\mathbf{k}$ og $\mathbf{B}_0$ og $\mathbf{B}_0$ er parallel med kryssproduktet av $\mathbf{k}$ og $\mathbf{E}_0$.

Vi har dermed vist at for at bølgefunksjonene skal tilfredsstille Maxwells likninger i tomrom, må de konstante vektorene $\mathbf{E}_0$, $\mathbf{B}_0$ og $\mathbf{k}$ være innbyrdes ortogonale.