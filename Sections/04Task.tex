\homeworkProblem[4] 

\subsection*{Faste stoffer}

Faste stoff er materialer som har en definert form og volum. De kan være krystallinske eller amorfiske, avhengig av den atomære strukturen. Krystallinske materialer har en periodisk, ordnet atomstruktur, mens amorfiske materialer har en mer tilfeldig atomstruktur. I faststoffelektronikk er det ofte krystallinske materialer som er av størst interesse på grunn av deres unike egenskaper.

I de kommende delene av presentasjonen vil vi se nærmere på ulike aspekter av krystallografi og resiprokt rom, og hvordan disse konseptene bidrar til å forstå og beskrive krystallinske materialer og deres egenskaper.

\subsection*{Krystaller, basis og gitter}

En krystall er et fast stoff der atomer, ioner eller molekyler er ordnet i et mønster som gjentar seg periodisk i tre dimensjoner. Den periodiske strukturen av en krystall er en direkte konsekvens av den ordnede plasseringen av partikler i materialet, og det er denne periodisiteten som gir krystaller deres unike egenskaper.

For å beskrive krystallstrukturen er det nyttig å bruke begrepene basis og gitter. En basis er en gruppe av atomer, ioner eller molekyler som danner en representativ enhet av krystallstrukturen. Gitteret, derimot, er et abstrakt matematisk konstruksjon som representerer den periodiske gjentagelsen av basisen i rommet. Gitteret kan betraktes som et sett med punkter som representerer posisjonene til basisen, og når basisen gjentas på hvert av disse punktene, dannes den fullstendige krystallstrukturen.

For å illustrere dette, kan vi tenke på et enkelt eksempel som et natriumklorid (NaCl) krystall. I dette tilfellet består basisen av to atomer, et natriumatom og et kloratom. Gitteret er et sett med punkter som representerer den periodiske gjentakelsen av basisen i rommet. Når basisen gjentas på hvert av disse punktene, dannes den krystallinske strukturen av natriumklorid.

Ved å forstå og beskrive krystallar, basis og gitter, kan vi få innsikt i krystallinske materialers egenskaper og hvordan de påvirkes av deres periodiske strukturer. I de neste delene av presentasjonen vil vi se nærmere på flere konsepter som er nødvendige for å beskrive og analysere krystallinske strukturer, inkludert einingsceller, Bravaisgitter og kubiske gitter.

\subsection*{Enhetsceller}

I dette avsnittet skal vi fokusere på einingsceller, som er en fundamental byggestein i krystallinske materialer og viktig for å forstå deres struktur og egenskaper. En einingscelle er den minste, repeterende enheten av et krystallgitter som inneholder hele krystallstrukturens mønster. Med andre ord, når en einingscelle gjentas periodisk i tre dimensjoner, dannes den fullstendige krystallstrukturen.

Einingsceller kan klassifiseres i henhold til deres geometri og symmetri. Det finnes syv krystallsystemer, som hver kan beskrives med en spesifikk type einingscelle: triklin, monoklin, ortorombisk, tetragonal, trigonal, heksagonal og kubisk. Disse krystallsystemene representerer forskjellige måter å organisere atomer i et krystallgitter og gir opphav til forskjellige krystallinske egenskaper.

Det er viktig å merke seg at einingscellen ikke nødvendigvis er den minste enheten som inneholder alle atomene i en krystallstruktur. Faktisk kan en einingscelle inneholde flere basiser, avhengig av krystallstrukturen. Når vi analyserer krystallinske materialer, er det ofte praktisk å velge den enkleste og mest symmetriske einingscellen som beskriver strukturen, men det kan også være situasjoner der det er nyttig å vurdere andre einingsceller.

For å oppsummere, einingsceller er en nøkkelkomponent i krystallstrukturen og gir oss et verktøy for å beskrive og analysere krystallinske materialer og deres egenskaper. I de neste delene av presentasjonen vil vi se nærmere på andre viktige konsepter i krystallografi, som Bravaisgitter, kubiske gitter og Miller indeksar, som alle hjelper oss med å forstå og beskrive krystallinske strukturer på en mer detaljert måte.

\subsection*{Bravaisgitter}

Nå skal vi se nærmere på et viktig konsept i krystallografi som hjelper oss med å klassifisere krystallstrukturer, nemlig Bravaisgitter. Et Bravaisgitter er et matematisk konstruksjon som representerer den periodiske gjentagelsen av punkter i rommet. Det finnes totalt 14 forskjellige Bravaisgitter som kan beskrive alle mulige krystallinske strukturer. Disse gitterene er delt inn i syv krystallsystemer som vi nevnte tidligere: triklin, monoklin, ortorombisk, tetragonal, trigonal, heksagonal og kubisk.

Bravaisgitterene er definert ved deres gitterparametere og symmetrier. Gitterparametrene inkluderer gitterkonstantene a, b og c, som representerer lengden av gittervektorene, samt vinklene $\alpha$, $\beta$ og $\gamma$ mellom dem. De 14 Bravaisgitterene er et resultat av alle kombinasjoner av gitterparametere og symmetrier som er mulige innenfor de syv krystallsystemene.

Et viktig aspekt ved Bravaisgitter er at de kun beskriver den geometriske gjentagelsen av punkter i rommet og ikke informasjon om basisen, som inneholder informasjon om atomene, ionene eller molekylene i krystallstrukturen. Når vi kombinerer et Bravaisgitter med en basis, får vi den fullstendige krystallstrukturen.

Ved å klassifisere krystallstrukturen ved hjelp av Bravaisgitter, kan vi lettere forstå og sammenligne egenskapene til forskjellige materialer. For eksempel kan vi identifisere likheter og forskjeller mellom materialer som har samme Bravaisgitter, men forskjellige basiser, og dermed få innsikt i hvordan atomære arrangementer påvirker deres egenskaper.

I de neste delene av presentasjonen vil vi se nærmere på kubiske gitter og deres varianter, samt Miller indeksar, som er viktige konsepter for å beskrive og analysere krystallinske strukturer og deres egenskaper.

\subsection*{Kubiske gitter (fcc, bcc, sc)}
I denne delen av presentasjonen skal vi se nærmere på kubiske gitter og deres tre varianter: face-centered cubic (fcc), body-centered cubic (bcc) og simple cubic (sc). Kubiske gitter tilhører det kubiske krystallsystemet og er spesielt viktige i faststoffelektronikk på grunn av deres symmetri og mange praktiske anvendelser.

La oss starte med simple cubic (sc) gitteret. I dette gitteret er atomene plassert i hvert hjørne av en kube. Gitterkonstantene a, b og c er like i lengde, og vinklene $\alpha$, $\beta$ og $\gamma$ er alle 90 grader. Simple cubic gitter er den enkleste typen kubisk gitter, men det finnes ikke mange materialer som naturlig har denne strukturen på grunn av lav pakkingseffektivitet.

Neste er body-centered cubic (bcc) gitteret. I tillegg til hjørneatomene har dette gitteret også et atom i sentrum av kuben. Bcc-gitteret har en høyere pakkingseffektivitet enn sc-gitteret og er typisk for metaller som jern, wolfram og krom.

Sist, men ikke minst, har vi face-centered cubic (fcc) gitteret. I dette gitteret er det atomer i hvert hjørne av kuben, samt i midten av hver av kubens seks flater. Fcc-gitteret har den høyeste pakkingseffektiviteten blant de kubiske gitterene og er typisk for metaller som aluminium, kobber og gull.

Betydningen av disse gittertypene for materialers egenskaper og bruksområder kan ikke understrekes nok. For eksempel er fcc- og bcc-gitterene ofte involvert i mekaniske egenskaper som duktilitet og hardhet, mens sc-gitteret kan være relevant for elektriske egenskaper som ledningsevne.

Ved å forstå og beskrive kubiske gitter og deres varianter, kan vi få innsikt i materialers egenskaper og hvordan de påvirkes av deres krystallinske strukturer. I de neste delene av presentasjonen vil vi se nærmere på Miller indeksar, Wigner-Seitz cella, og reelt og resiprokt gitter, som alle hjelper oss med å forstå og beskrive krystallinske strukturer og deres egenskaper på en mer detaljert måte.

\subsection*{Miller indekser}

I dette avsnittet skal vi introdusere Miller indeksar, som er et viktig verktøy i krystallografi for å beskrive krystallplan og retninger i krystallinske materialer. Miller indeksar er en notasjon som består av tre heltall (h, k, l) som er proporsjonale med de resiproke av gitterkonstantene a, b og c langs de tre krystallakser.

Miller indeksar brukes til å identifisere og beskrive spesifikke krystallplan i en krystallstruktur. Krystallplan er viktige fordi de påvirker mange fysiske egenskaper, som mekaniske, optiske og elektriske egenskaper. For eksempel, i halvledermaterialer som silisium, kan forskjellige krystallplan påvirke egenskaper som ledningsevne og båndstruktur.

For å beregne Miller indeksar for et krystallplan, kan vi følge disse trinnene:

Finn punktene der krystallplanet krysser krystallaksene.
\begin{enumerate}
    \item Finn punktene der krystallplanet krysser krystallaksene.
    \item Ta de resiproke verdiene av disse punktene.
    \item Finn minste felles multiplum for å gjøre verdiene til hele tall.
    \item Skriv indeksene som (h, k, l).

\end{enumerate}
For eksempel, i et kubisk krystall med gitterkonstant a, hvis et krystallplan krysser x-aksen ved a, y-aksen ved a og z-aksen ved a, vil de resiproke verdiene være 1, 1 og 1. Minste felles multiplum er 1, så Miller indeksar for dette planet blir (1, 1, 1).

I tillegg til krystallplan, kan Miller indeksar også brukes til å beskrive retninger i krystallinske materialer. For å representere en retning, bruker vi en lignende notasjon, men med vinkelparenteser: ⟨h, k, l⟩. Retningene er viktige for å beskrive fenomener som dislokasjoner og spredning av stråling i krystaller.

Miller indeksar gir oss en enkel og konsistent måte å beskrive og sammenligne krystallplan og retninger i forskjellige materialer. Ved å forstå og bruke Miller indeksar, kan vi få bedre innsikt i hvordan krystallinske strukturer og deres egenskaper er relatert. I de neste delene av presentasjonen vil vi se nærmere på Wigner-Seitz cella, samt reelt og resiprokt gitter, som er viktige konsepter for å forstå og beskrive krystallinske strukturer og deres egenskaper.

\subsection*{Wigner-Seitz cella}

I denne delen av presentasjonen vil vi introdusere Wigner-Seitz cella, et annet nyttig konsept i krystallografi som hjelper oss med å forstå og beskrive krystallinske strukturer. Wigner-Seitz cella er en spesiell type einingscelle som er konstruert rundt et gitterpunkt og inneholder det punktet og alle punktene nærmere det enn noen annet gitterpunkt.

For å konstruere en Wigner-Seitz celle, følg disse trinnene:

\begin{enumerate}
    \item Velg et gitterpunkt som sentrum av cellen.
    \item Tegn linjer fra det valgte gitterpunktet til alle nærmeste naboer.
    \item Tegn midtplanene mellom gitterpunktet og hver nabo.
    \item Avgrens cellen ved skjæringspunktene mellom midtplanene.
\end{enumerate}

Wigner-Seitz cella har en spesiell egenskap: den inneholder alle symmetriene til det underliggende gitteret og er derfor en representativ enhet for krystallstrukturen. Den gir oss et intuitivt og geometrisk bilde av hvordan atomer, ioner eller molekyler er arrangert i krystallet og deres innbyrdes avstander.

Et viktig konsept relatert til Wigner-Seitz cella er det resiproke gitteret, som vi vil se på i neste del av presentasjonen. Resiprokt gitter er en annen matematisk konstruksjon som hjelper oss med å beskrive krystallinske strukturer og deres egenskaper, spesielt i forbindelse med spredning av bølger og partikler.

Ved å forstå og bruke Wigner-Seitz cella i vår analyse av krystallinske materialer, kan vi få bedre innsikt i deres strukturer og egenskaper, samt deres interaksjon med bølger og partikler. Dette er spesielt viktig i faststoffelektronikk, der krystallstrukturen og dens egenskaper har en direkte innvirkning på materialers elektriske, optiske og mekaniske egenskaper.

\subsection*{Reelt og resiprokt gitter}
Til slutt i presentasjonen vil vi se på reelt og resiprokt gitter, to viktige konsepter i krystallografi som hjelper oss med å beskrive og analysere krystallinske strukturer og deres egenskaper.

Reelt gitter er det romlige arrangementet av punkter som representerer periodiske gjentakelser av atomer, ioner eller molekyler i et krystallinsk materiale. Vi har allerede diskutert forskjellige typer reelle gitter, som Bravaisgitter og kubiske gitter. Reelt gitter gir oss en geometrisk beskrivelse av krystallstrukturen og er nyttig for å forstå materialets symmetri og pakkingseffektivitet.

Resiprokt gitter er en matematisk konstruksjon som er relatert til det reelle gitteret, men beskriver materialets egenskaper i en annen måte. I stedet for å representere den fysiske plasseringen av atomer, ioner eller molekyler, representerer det resiproke gitteret bølgevektorer som er forbundet med periodiske egenskaper av krystallet. Resiprokt gitter er spesielt viktig for å forstå og beskrive fenomener som spredning av bølger og partikler, som røntgen- eller elektrondiffraksjon, i krystallinske materialer.

For å konstruere det resiproke gitteret fra det reelle gitteret, kan vi bruke følgende trinn:

Finn de primitive gittervektorene i det reelle gitteret, merket som a, b og c.

Ved å analysere det resiproke gitteret, kan vi få innsikt i materialets egenskaper, som båndstruktur, gitterdynamikk og spredning av stråling. Dette er spesielt viktig i faststoffelektronikk, der krystallstrukturen og dens egenskaper har en direkte innvirkning på materialers elektriske, optiske og mekaniske egenskaper.