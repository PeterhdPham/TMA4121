\homeworkProblem[3]

\subsection*{Kvantebrønn i 3D (partikkel i boks)}

I en 3D kvantebrønn har vi tre kvantetall, $n_x$, $n_y$, og $n_z$, som hver representerer partikkelens energinivå i de respektive retningene. Disse kvantetallene er positive heltall og kan ikke være null.

Energien til en partikkel i en 3D kvantebrønn kan uttrykkes som:

\begin{equation}
E_{n_x, n_y, n_z} = \frac{\hbar^2 \pi^2}{2m} \left( \frac{n_x^2}{L_x^2} + \frac{n_y^2}{L_y^2} + \frac{n_z^2}{L_z^2} \right)
\end{equation}

Når vi snakker om atomer, introduseres andre kvantetall for å beskrive elektronenes tilstander. De tre kvantetallene er hovedkvantetallet ($n$), banespinnkvantetallet ($l$) og det magnetiske kvantetallet ($m$). Disse kvantetallene spiller en viktig rolle i bestemmelsen av elektronenes energinivåer og deres romlige fordeling rundt atomkjernen.

Hovedkvantetallet $n$ er et positivt heltall og bestemmer energinivået til elektronet. Større verdier av $n$ representerer høyere energinivåer og større avstand fra kjernen.

Banespinnkvantetallet $l$ er et heltall som varierer fra 0 til $n-1$. Dette kvantetallet gir informasjon om formen på elektronets bane rundt kjernen og angis ofte med bokstavene s, p, d og f, som tilsvarer $l$ = 0, 1, 2 og 3.

Det magnetiske kvantetallet $m$ er et heltall som varierer fra $-l$ til $+l$. Dette kvantetallet beskriver orienteringen av elektronets bane i forhold til et ytre magnetfelt og påvirker også energinivåene til elektronet når det er et magnetfelt til stede.

\subsection*{Degenerasjon}

Degenerasjon refererer til fenomenet der flere tilstander i et kvantesystem har samme energi. Når et system har degenererte energinivåer, betyr det at det finnes flere tilstander som kan beskrive systemet uten å endre dets energi. Dette skjer ofte i kvantemekaniske systemer som kvantebrønner.

\subsection*{Elektronkonfigurasjon}
Elektronkonfigurasjonen til et atom beskriver fordelingen av elektroner i atomets energinivåer og orbitaler. Det er avgjørende for å forstå atomers kjemiske egenskaper og hvordan de danner bindinger med andre atomer. To viktige regler som hjelper oss med å bestemme elektronkonfigurasjonen er Madelungs regel og Hund's regel.

\textbf{Madelungs regel} gir en retningslinje for rekkefølgen av elektronfylling i orbitaler. Ifølge denne regelen fylles orbitaler med lavere energi først. Madelungs regel kan summeres opp ved hjelp av regelen $(n+l)$, der orbitaler med lavere $(n+l)$-verdier fylles først. Hvis to orbitaler har samme $(n+l)$-verdi, fylles orbitaler med lavere $n$-verdi først.

For eksempel fylles 1s-orbitaler før 2s-orbitaler fordi $(n+l)$-verdien er lavere for 1s (1+0) enn for 2s (2+0). På samme måte fylles 2s før 2p, og 3s fylles før 3p og 3d. Denne regelen kan brukes til å finne elektronkonfigurasjonen til ethvert grunnstoff i det periodiske systemet.

\textbf{Hund's regel} gir retningslinjer for å bestemme elektronenes spinnkonfigurasjon når de fyller orbitaler med samme energi, som de degenererte orbitalene i p-, d- og f-blokkene. Hund's regel kan oppsummeres i tre hovedpunkter:
\begin{itemize}
    \item Elektroner vil først fylle hvert orbital i en degenerert gruppe med samme spinn (enten opp eller ned) før de begynner å pare seg med motsatt spinn.
    \item Et atom vil ha maksimalt totalt spinnmoment (summen av elektronenes individuelle spinn) som er tillatt av Paulis eksklusjonsprinsipp for en gitt elektronkonfigurasjon.
    \item For atomer med mindre enn halvfullt fylte orbitaler, vil den elektronkonfigurasjonen med det laveste totale orbitalmomentet være mest stabil.
\end{itemize}


\subsection*{Bindinger i faste stoffer}
I faste stoffer holdes atomer sammen av forskjellige typer kjemiske bindinger. Disse bindingene bestemmer mange av stoffets egenskaper, som elektrisk ledningsevne. Det er fire hovedtyper av bindinger i faste stoffer: ioniske, kovalente, metalliske og van der Waals-bindinger.

\textbf{Ioniske bindinger:} Ioniske bindinger oppstår mellom metaller og ikke-metaller. De dannes når et metallatom overfører et eller flere elektroner til et ikke-metallatom, noe som resulterer i dannelsen av ioner med motsatte ladninger. Disse ionene tiltrekker hverandre på grunn av elektrostatisk tiltrekning og danner et ionisk fast stoff. Ioniske forbindelser har generelt høy smelte- og kokepunkt, og de leder elektrisitet når de er i smeltet eller oppløst tilstand.

\textbf{Kovalente bindinger:} Kovalente bindinger dannes mellom ikke-metallatomer. De oppstår når to eller flere atomer deler elektroner for å oppnå en stabil elektronkonfigurasjon. Kovalente bindinger kan være enkle, doble eller triple, avhengig av antall elektronpar som deles mellom atomene. Kovalente forbindelser kan danne molekylære faste stoffer, der molekylene holdes sammen av svake van der Waals-bindinger, eller nettverksfaststoffer, der kovalente bindinger danner et tredimensjonalt nettverk.

\textbf{Metalliske bindinger:} Metalliske bindinger oppstår mellom metaller. De er resultatet av delingen av valenselektroner mellom metallatomer, som skaper et "elektronsjø" der elektronene kan bevege seg fritt. Denne frie bevegelsen av elektroner gir metallene deres karakteristiske egenskaper, som høy elektrisk og termisk ledningsevne, glans og duktilitet.

\textbf{Van der Waals-bindinger:} Van der Waals-bindinger er svake intermolekylære krefter som oppstår mellom molekyler eller atomer på grunn av midlertidige ladningsfordelinger. Disse kreftene inkluderer dipol-dipol-interaksjoner, dipol-induserte dipolinteraksjoner og London-dispersjonskrefter. Van der Waals-bindinger er svakere enn ioniske, kovalente og metalliske bindinger, og de er ansvarlige for å holde molekyler sammen i molekylære faste stoffer.

\subsection*{Dobbel kvantebrønn i faste stoffer}

I faste stoffer kan doble kvantebrønner oppstå når to atomer, som hydrogenatomer, kommer nær hverandre og danner bindinger. I dette tilfellet vil de individuelle atomene ha deres egne kvantebrønner for elektronene, men når de kommer nærmere hverandre, vil deres kvantebrønner overlappe og danne et nytt system av kvantebrønner. Dette systemet kan analyseres i form av bonding orbitaler og anti-bonding orbitaler.

Bonding orbitaler oppstår når to atomer nærmer seg hverandre, og deres elektronbølgefunksjoner har samme fase. Dette fører til en konstruktiv interferens mellom bølgefunksjonene, noe som resulterer i økt elektrontetthet mellom atomene. Denne økte elektrontettheten skaper en tiltrekkende kraft mellom atomene og stabiliserer bindingen. Bonding orbitaler er derfor lavere i energi enn de isolerte atomorbitalene.

Anti-bonding orbitaler oppstår derimot når de overlappende elektronbølgefunksjonene har motsatte faser. Dette fører til destruktiv interferens og en reduksjon i elektrontetthet mellom atomene. Den reduserte elektrontettheten skaper en frastøtende kraft mellom atomene og destabiliserer bindingen. Anti-bonding orbitaler er høyere i energi enn de isolerte atomorbitalene og bonding orbitalene.

Når det gjelder hydrogenbindinger, vil to hydrogenatomer danne en binding når deres elektronbølgefunksjoner overlapper på en slik måte at det skapes et bonding orbital. Når elektronene befinner seg i bonding orbitalen, vil hydrogenatomene tiltrekkes av hverandre og danne en stabil kovalent binding.

Avstanden mellom kvantebrønnene er relatert til avstanden mellom atomene i det faste stoffet. Når avstanden mellom atomene er stor, vil deres kvantebrønner ha liten overlapping og deres elektroner vil i stor grad være begrenset til deres respektive atomer. Når avstanden mellom atomene reduseres, vil overlappingen av deres kvantebrønner øke, noe som fører til økt mulighet for elektronene å bevege seg mellom de to atomene. Dette prinsippet om overlapping og superposisjon av bølgefunksjoner er avgjørende for å forstå bindinger og elektroniske egenskaper i faste stoffer.